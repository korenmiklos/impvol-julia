%2multibyte Version: 5.50.0.2960 CodePage: 65001

\documentclass[12pt]{article}
%%%%%%%%%%%%%%%%%%%%%%%%%%%%%%%%%%%%%%%%%%%%%%%%%%%%%%%%%%%%%%%%%%%%%%%%%%%%%%%%%%%%%%%%%%%%%%%%%%%%%%%%%%%%%%%%%%%%%%%%%%%%%%%%%%%%%%%%%%%%%%%%%%%%%%%%%%%%%%%%%%%%%%%%%%%%%%%%%%%%%%%%%%%%%%%%%%%%%%%%%%%%%%%%%%%%%%%%%%%%%%%%%%%%%%%%%%%%%%%%%%%%%%%%%%%%
\usepackage{amssymb}
\usepackage{amsmath}
\usepackage{setspace}
\usepackage{geometry}
\usepackage{fancyhdr}
\usepackage{harvard}
\usepackage{sectsty}
\usepackage{endnotes}
\usepackage{graphicx}
\usepackage{float}

\setcounter{MaxMatrixCols}{10}
%TCIDATA{OutputFilter=LATEX.DLL}
%TCIDATA{Version=5.50.0.2960}
%TCIDATA{Codepage=65001}
%TCIDATA{<META NAME="SaveForMode" CONTENT="1">}
%TCIDATA{BibliographyScheme=Manual}
%TCIDATA{LastRevised=Thursday, June 06, 2019 18:01:13}
%TCIDATA{<META NAME="GraphicsSave" CONTENT="32">}
%TCIDATA{Language=American English}

\setlength{\footnotesep}{11.0pt}
\renewcommand{\baselinestretch}{2}
\newtheorem{theorem}{Theorem}
\newtheorem{acknowledgement}[theorem]{Acknowledgement}
\newtheorem{algorithm}[theorem]{Algorithm}
\newtheorem{axiom}[theorem]{Axiom}
\newtheorem{case}[theorem]{Case}
\newtheorem{claim}[theorem]{Claim}
\newtheorem{conclusion}[theorem]{Conclusion}
\newtheorem{condition}[theorem]{Condition}
\newtheorem{conjecture}[theorem]{Conjecture}
\newtheorem{corollary}[theorem]{Corollary}
\newtheorem{criterion}[theorem]{Criterion}
\newtheorem{definition}[theorem]{Definition}
\newtheorem{example}[theorem]{Example}
\newtheorem{exercise}[theorem]{Exercise}
\newtheorem{lemma}[theorem]{Lemma}
\newtheorem{notation}[theorem]{Notation}
\newtheorem{problem}[theorem]{Problem}
\newtheorem{proposition}[theorem]{Proposition}
\newtheorem{remark}[theorem]{Remark}
\newtheorem{solution}[theorem]{Solution}
\newtheorem{summary}[theorem]{Summary}
\newenvironment{proof}[1][Proof]{\textbf{#1.} }{\ \rule{0.5em}{0.5em}}
\geometry{left=1in,right=1in,top=1in,bottom=1in}
\renewcommand{\thesection}{\Roman{section}}
\renewcommand{\thesubsection}{\thesection.\Alph{subsection}}
\graphicspath{{C:/Users/tenreyro/Dropbox/impvol/Finalversion/impvol_text/}}
\input{tcilatex}
\begin{document}

\title{Diversification through Trade \thanks{%
Thanks: Referees, Pol Antras, Costas Arkolakis, Robert Barro, Fernando
Broner, Ariel Burstein, Lorenzo Caliendo, Julian DiGiovanni, Bernardo
Guimaraes, Nobu Kiyotaki, Pete Klenow, David Laibson, Fabrizio Perri, Steve
Redding, Ina Simonovska, Jaume Ventura, Romain Wacziarg, and seminar
participants at Bocconi, Birmingham, CREI, Princeton, Penn, Yale, NYU, UCL,
LBS, LSE, Toulouse, Warwick, as well as participants at SED, ESSIM, and the
Nottingham trade conference. Calin Vlad Demian, Balazs Kertesz, Federico
Rossi, and Peter Zsohar provided superb research assistance. Caselli
acknowledges financial support from the Luverlhume Fellowship. Koren
acknowledges financial support from the European Research Council (ERC)
starting grant 313164. Tenreyro acknowledges financial support from the ERC
starting grant 240852. \newline
$^{\dag }$ London School of Economics, CfM, CEPR. $^{\ddag }$ Central
European University, MTA KRTK, CEPR. $^{\S }$ European Comission.
Correspondence: s.tenreyro@LSE.ac.uk.} }
\author{Francesco Caselli$^{\dag }$ \and Miklos Koren$^{\ddag }$ \and Milan
Lisicky$^{\S }$ \and Silvana Tenreyro$^{\dag }$}
\date{This draft: November 2017}
\maketitle

\begin{abstract}
A widely held view is that openness to international trade leads to higher
income volatility, as trade increases specialization and hence exposure to
sector-specific shocks. Contrary to this common wisdom, we argue that when
country-wide shocks are important, openness to international trade can lower
income volatility by reducing exposure to domestic shocks, and allowing
countries to diversify the sources of demand and supply across countries.
Using a quantitative model of trade, we assess the importance of the two
mechanisms (sectoral specialization and cross-country diversification) and
show that in recent decades international trade has reduced economic
volatility for most countries.
\end{abstract}

\section{Introduction}

An important question at the crossroads of macro-development and
international economics is whether and how openness to trade affects
macroeconomic volatility. A widely held view in academic and policy
discussions, which can be traced back at least to Newbery and Stiglitz
(1984), is that openness to international trade leads to higher income
volatility. The origins of this view are rooted in a large class of theories
of international trade predicting that openness to trade increases
specialization. Because specialization in production tends to increase a
country's exposure to shocks specific to the sectors (or range of products)
in which the country specializes, it is generally inferred that trade
increases volatility.\ This view seems present in policy circles, where
trade openness is often perceived as posing a trade-off between the first
and second moments (i.e., trade causes higher productivity at the cost of
higher volatility).\footnote{%
See for example the report on \textquotedblleft Economic openness and
economic prosperity: trade and investment analytical
paper\textquotedblright\ (2011), prepared by the U.K. Department of
International Development.}

This paper revisits the common wisdom on two conceptual grounds. First, the
existing wisdom is strongly predicated on the assumption that
sector-specific shocks (hitting a particular sector) are the dominant source
of income volatility. The evidence, however, does not support this
assumption. Indeed, country-specific shocks (shocks common to all sectors in
a given country) are at least as important as sector-specific shocks in
shaping countries' volatility patterns (e.g. Stockman, 1988, Costello, 1993,
Koren and Tenreyro, 2007).\footnote{%
Both Stockman and Costello find that country-specific shocks are more
important than sector-specific shocks in shaping volatility patterns in
seven (resp., five) industrialized countries. Using a wider sample of
countries and a different method, Koren and Tenreyro confirm these results,
and find that the relative weight of country-specific shocks is even more
relevant in less developed economies.} The first contribution of this paper
is to show analytically that when country-specific shocks are an important
source of volatility, openness to international trade can lower income
volatility. In particular, openness reduces a country's exposure to domestic
shocks, and allows it to diversify its sources of demand and supply, leading
to potentially lower overall volatility. This is true as long as the
volatility of shocks affecting trading partners is not too large, or the
covariance of shocks across countries is not too large. In other words, we
show that the sign and size of the effect of openness on volatility depends
on the variances and covariances of shocks across countries.

The paper furthermore questions the mechanical assumption that higher
sectorial specialization per se leads to higher volatility. Indeed, whether
income volatility increases or decreases with specialization depends on the
intrinsic volatility of the sectors in which the economy specializes in, as
well as on the covariance among sectorial shocks and between sectorial and
country-wide shocks.

We make these points in the context of a quantitative, multi-sector,
stochastic model of trade and GDP determination. The model builds on a
variation of Eaton and Kortum (2002), Alvarez and Lucas (2007), and Caliendo
and Parro (2015), augmented to allow for country-specific and
sector-specific shocks.\footnote{\label{FTlit}Variations of this model have
been used to address a number of questions in international economics. An
incomplete list includes Hsieh and Ossa (2011) and di Giovanni, Levchenko,
and Zhang (2014), who study the global welfare impact of China's trade
integration and technological change; Levchenko and Zhang (2013), who
investigate the impact of trade with emerging countries on labour markets;
Burstein and Vogel (2016) and Parro (2013), who study the effect of
international trade on the skill premium; Caliendo, Parro, Rossi-Hansberg
and Sarte (2014), who study the impact of regional productivity changes on
the U.S. economy, and so on. None of these applications, however, focuses on
the impact of openness to trade on volatility. A partial exception is
Burgess and Donaldson (2012), which we discuss below.} In each sector,
production combines equipped labour with a variety of tradable inputs.
Producers source tradable inputs from the lowest-cost supplier (where supply
costs depend on the supplier's productivity as well as trade costs), after
productivity shocks have been realized. This generates the potential for
trade to \textquotedblleft insure\textquotedblright\ against shocks, as
producers can redirect input demand to countries experiencing positive
supply shocks. However, (equipped) labor must be allocated to sectors before
productivity shocks are realized. This friction allows us to capture the
traditional specialization channel, because it reduces a country's ability
to respond to sectorial shocks by reallocating resources to other sectors.
An extension of the model allows for ex-post sectorial reallocation of
equipped labour in the presence of reallocation costs.

We use the model in conjunction with sector-level production and bilateral
trade data for a diverse group of 24 countries to quantitatively assess how
changes in trading costs since the early 1970s have affected income
volatility.\footnote{%
The data are disaggregated into 24 sectors. We stop the analysis in 2007 as
our model abstracts from the factors underlying the financial crisis.} We
find that the decline in trade costs since the 1970s has caused sizeable
reductions in income volatility in the vast majority of\textbf{\ }the
countries in our sample. The range of changes in volatility due to trade
varies significantly across countries, with the largest reductions being
nearly 80\%. On average, volatility fell 37\% compared to a counterfactual
where trade barriers remain at their early-1970s level.

The general decline in volatility due to trade is the net result of the two
different mechanisms discussed above: sectorial specialization, and
country-wide diversification. The country-wide diversification mechanism
again contributed to lower volatility in most of the countries in our
sample, consistent with our key idea that trade is a source of
diversification of country-wide shocks. The sectoral-specialization
mechanism increased volatility in 54\% of the countries in the sample.
Consistent with our theoretical points above, then, the common wisdom that
specialization leads to greater volatility fails to apply almost as often as
it does. The crucial and most important point, however, is that the
country-wide diversification effect is on average eight times as large as
the sectoral-specialization effect, so that the net effect is that trade
reduces volatility in the overwhelming majority of cases.\footnote{%
While in this paper the focus is on country-wide shocks and sectoral shocks,
future work could even extend to \textquotedblleft
granular\textquotedblright\ shocks. See, e.g. di Giovanni and Levchenko
(2012) for a discussion of trade and volatility with granular shocks and di
Giovanni, Levchenko and Mejean (2014) for a country-level application.}

We subject our results to a variety of robustness checks and extensions. In
the latter, we find that it is important to feature a detailed input-output
structure to fully capture the impact of trade on volatility. We also find
that the impact of trade on volatility is not driven by the emergence of
China, but it is a much more general phenomenon.

The focus of our quantitative evaluation is real income, defined as nominal
GDP deflated by a cost of living index. In the model, the cost of living
index is a preferenced-based ideal price deflator. In the data counterpart,
the cost of living index is the CPI. Hence, we work with a welfare-relevant
notion of income.\footnote{%
Kehoe and Ruhl (2008) and Burstein and Cravino (2015) study the theoretical
impact of foreign productivity shocks on various measures of domesitc
economic activity. In general, foreign productivity shocks (or other sources
of change in the terms of trade) have little first-order effects on
production-based measures of activity (e.g. GDP deflated by the GDP
deflator), while they have first-order effects on welfare-based measures.}
Indeed, we could alternatively have focused on consumption volatility,
because, for most countries in the world, income and consumption
fluctuations are almost perfectly correlated, and in our model income equals
consumption. For the same reason\textbf{, }our model abstracts from trade in
financial assets.

The fact that openness to trade has ambiguous predicted effects on
volatility might partly explain why direct empirical evidence on the effect
of openness on volatility has yielded mixed results. Some studies find that
trade decreases volatility [e.g., Bejan (2006), Buch, D\"{o}pke and
Strotmann (2009), Cavallo (2008), Haddad, Lim and Saborowski (2010),
Parinduri (2011), Burgess and Donaldson (2012)], while others find that
trade increases it [e.g., Rodrik (1998), Easterly, Islam, and Stiglitz
(2001), Kose, Prasad, and Terrones (2003), di Giovanni and Levchenko
(2009)]. The model-based analysis can circumvent the problem of causal
identification faced by many empirical studies, allowing for counterfactual
exercises that isolate the effect of trade costs on volatility. Moreover, it
can cope with highly heterogenous trade effects across countries.

Besides contrasting with assessments of the trade-volatility relationship
based on (a simplistic understanding of) the specialization framework, our
paper also offers an alternative perspective on openness and volatility to
the so-called International Real Business Cycle approach. Backus, Kehoe, and
Kydland (1992) show that GDP volatility is higher in the open economy than
in the closed economy, as capital inputs are allocated to production in the
country with the most favorable technology shock. Hence, income fluctuations
are amplified in an open economy. In our multi-country, multi-sector
setting, instead, income volatility can---and often does---decrease with
openness, as intra-temporal trade in inputs allows countries with less
favorable productivity shocks to source inputs from abroad, thus reducing
income (as well as consumption) volatility.\footnote{%
Also related is the empirical literature initiated by Frankel and Rose
(1998), who documented a strong correlation between bilateral trade flows
and GDP comovements between pairs of countries (see also, e.g., Kose and Yi
(2001), Arkolakis and Ramanarayanan (2009)). Our main focus in this paper is
on the effect of trade on \textit{volatility}---and the channels mediating
this effect---but the quantitative approach we follow in our counterfactual
exercise can potentially be extended to also identify the effect of trade on
bilateral comovement---and indeed, other higher-order moments.}

A paper that is closely related to ours is Burgess and Donaldson (2012), who
use the Eaton-Kortum model in conjunction with data on the expansion of
railroads across regions in India to assess whether real income became more
or less sensitive to rainfall shocks, as India's regions became more open to
trade. The authors find that the decline in transportation costs lowered the
impact of productivity shocks on real income, implying a reduction in
volatility. Our analysis is at a higher level of generality,\textbf{\ }and
highlights that, while a reduction in volatility has been experienced by
many countries as they became more open to trade, the size and sign of the
trade effect on volatility may be--- and indeed has been---different across
different countries.\footnote{%
See also Donaldson (2015), where the question is also addressed in the
context of India's railroad expansion. There is\ also a growing literature
on the effect of globalization on income risk and inequality. We do not
focus on distributional effects within countries in this paper, though it is
obviously a very important issue, and a natural next step in our research.
For theoretical developments in that area, see for example, Anderson (2011)
and the references therein.}

\textbf{While this paper focuses on contrasting our new
diversification-through-trade mechanism with the traditional sectoral
specialization mechanism, it leaves to future work to include the role of
\textquotedblleft granular\textquotedblright\ shocks. As pointed out by di
Giovanni and Levchenko (2012), if openness to trade increases concentration
the impact of granular shocks is exacerbated, potentially leading to an
increase in volatility. See also di Giovanni, Levchenko and Mejean (2014)
for a country-level application.}

The remainder of the paper is organized as follows. Section II presents the
model and solves analytically for two special cases, autarky and costless
free trade. Section III introduces the data and calibration. Section IV
presents the quantitative results, including robustness checks and
extensions. Section V presents concluding remarks. The Appendix contains
further derivations and a detailed description of the datasets used in the
paper.

\section{A Model of Trade with Stochastic Shocks}

The baseline model builds on a multi-sector variation of Eaton and Kortum
(2002), Alvarez and Lucas (2006), and Caliendo and Parro (2015), augmented
to allow for stochastic shocks, as well as frictions to the allocation of
non-produced (and non-traded) inputs across sectors.

\subsection{Model Assumptions}

The world economy is composed of $N$ countries. In each country $n$ there is
a final consumption good. The consumption good is a bundle of sectoral goods
produced by $J$ sectors. In turn, each sectoral output is a bundle of
sector-specific varieties. Each sectoral variety can be produced
domestically or imported. Domestic production of sectoral varieties uses
non-produced inputs, to which we refer as \textquotedblleft equipped
labor,\textquotedblright\ and other sectoral goods acting as intermediates.
All markets are perfectly competitive.

The consumption bundle $C_{nt}$ is packaged by a consumption-good producer
using the Cobb-Douglas aggregate 
\begin{equation}
C_{nt}=\prod\nolimits_{j=1}^{J}\left( C_{nt}^{j}\right) ^{\alpha _{t}^{j}},
\label{aggregate}
\end{equation}%
where $C_{nt}^{j}$ is the quantity of sectoral good $j$ used for
consumption, and $\sum_{j=1}^{J}\alpha _{t}^{j}=1$. The $\alpha $s are
allowed to change over time to capture possible changes in tastes. 
In Section \ref{Sces} we investigate the robustness of our results to a
specification of preferences in which the elasticity of substitution among
sectoral goods is not unitary.

Sectoral output in sector $j$, $Q_{nt}^{j}$, is 
\begin{equation}
Q_{nt}^{j}=\left[ \int_{0}^{1}q_{nt}(\omega ^{j})^{\frac{\eta -1}{\eta }%
}d\omega ^{j}\right] ^{\frac{\eta }{\eta -1}},  \label{sectoroutput}
\end{equation}%
where $q_{nt}(\omega ^{j})$ is the quantity of sectoral variety $\omega ^{j}$
used in sector $j$, and $\eta >0$ is the elasticity of substitution across
goods within a given sector. Implicit in this formulation is the assumption
that each sector relies on a continuum of sector-specific varieties, $\omega
^{j}$.

The technology for producing good $\omega ^{j}$ in country $n$ is 
\begin{equation}
x_{nt}(\omega ^{j})=A_{nt}^{j}z_{n}(\omega ^{j})l_{nt}(\omega ^{j})^{\beta
^{j}}\prod\nolimits_{k=1}^{J}M_{nt}^{k}(\omega ^{j})^{\gamma ^{kj}},
\label{eqinput}
\end{equation}%
where $x_{nt}(\omega ^{j})$ is the output of good $\omega ^{j}$ by country $%
n $ at time $t$; $M_{nt}^{k}(\omega ^{j})$ is the amount of sector $k$
output used by country $n$ in the production of good $\omega ^{j};$ $%
l_{nt}(\omega ^{j})$ is the corresponding amount of equipped labour; $%
z_{n}(\omega ^{j})$ is a time-invariant variety-specific productivity
factor; and $A_{nt}^{j}$ is a time-varying productivity shock common to all
the varieties in sector $j $. The exponent $\gamma ^{kj}$ captures the share
of sector $k$ in the total production cost of sector $j.$ We assume constant
returns to scale, or $\beta ^{j}+\sum_{k=1}^{J}\gamma ^{kj}=1$, for all $j$.
Notice that (\ref{eqinput}) allows for a rich input-output structure, as the
intensity with which each sector's output is used as intermediate by other
sectors varies across all sector pairs.

Building on the literature, we assume the productivities $z_{n}(\omega ^{j})$
follow a sector-specific, time-invariant Fr\'{e}chet distribution $%
F_{n}^{j}(z)=\exp (-T_{n}^{j}z^{-\theta })$. A higher $T_{n}^{j}$ shifts the
distribution of productivities to the right, that is leading to
probabilistically higher productivities. A higher $\theta $ decreases the
dispersion of the productivity distribution, and hence reduces the scope for
comparative advantage. The $z$\ terms are the main determinants of long-term
comparative advantage in our model.

The shocks to $A_{nt}^{j}$ over time are interpreted as standard TFP shocks,
and are what make the model stochastic at the aggregate level. We will later
decompose them into a country-specific component and a sector-specific
component. This decomposition will be used to identify separately the
country diversification and the sectoral specialization channels.

The intermediate goods $\omega ^{j}$ can be produced locally or imported
from other countries. Delivering a good from country $n$ to country $m$ in
sector $j$ and time period $t$ results in $0<\kappa _{mnt}^{j}\leq 1$ goods
arriving at $m$; we assume that $\kappa _{mnt}^{j}\geq \kappa
_{mkt}^{j}\kappa _{knt}^{j}\quad \forall m,n,k,j,t$ and $\kappa _{nnt}^{j}=1$%
. All costs incurred are net losses.\footnote{%
In the calibration, the $\kappa $s will reflect all trading costs, including
tariffs; so implicitly we adopt the extreme assumption that tariff revenues
are wasted---or at least not rebated back to agents in a way that would
interact with the allocation of resources in the economy.} Under the
assumption of perfect competition, goods are sourced from the lowest-cost
producer, after adjusting for transport costs. The sectoral outputs $%
Q_{nt}^{j}$ are nontraded.

At a given point in time $t$, country $n$ is endowed with $L_{nt}$ units of
a primary (non produced) input, which we interpret as equipped labour. At
the beginning of each period, before the realization of the shocks $%
A_{nt}^{j}$, a representative consumer decides on the optimal allocation of
the primary input $L_{nt}$ across the different sectors, $L_{nt}^{j}$. After
the shocks to productivity are realized, equipped labour can be reallocated
within a sector, but not across sectors. Next, production and consumption
take place. Clearly, clearing in the input market within a sector implies 
\begin{equation*}
L_{nt}^{j}=\int_{0}^{1}l_{nt}(\omega ^{j})d\omega ^{j}.
\end{equation*}

The lack of ex-post reallocation across sectors in a given period aims at
capturing the idea that in the short run it is costly to reallocate
productive factors across sectors. Aside from realism, our main intention in
including it is that we wish to nest into our model the traditional view
that trade causes volatility by pushing countries to specialize - thus
making them overly responsive to sectoral shocks. Without frictions to
sectoral reallocation, this mechanism could not arise, as the economy would
respond to shocks by moving labor from the negatively-affected sectors to
the sectors receiving (relatively) positive shocks. Our model would then
feature only our novel mechanism, namely the diversification of
country-level shocks.\footnote{%
In the quantification, a period will be one year. This amounts to assuming
that it takes at least one year for resources to be reallocated across
sectors. In Section V we relax the assumption of full rigidity within one
period, and allow for ex post sectoral reallocation of equipped labour
subject to an adjustment cost, which we calibrate to match sectoral
reallocation flows in the data.}

The representative agent has a per-period utility flow $\log (C_{nt})$.%
\footnote{%
The log utility assumption gives rise to a particularly intutive and
tractable decision rule for the labor allocation.} Because there is no
(endogenous) intertemporal trade and no capital in the economy, the only
decision the representative agent has to take in each period is the
allocation of equipped labor across sectors before observing the shock
realizations. Since labor can be freely reallocated at the beginning of each
period, this is a purely static decision.

Since equipped labor is the only non-produced input, the per period budget
constraint in each period is: 
\begin{equation}
P_{nt}C_{nt}=\sum\nolimits_{j=1}^{J}w_{nt}^{j}L_{nt}^{j}  \label{BC}
\end{equation}%
where $P_{nt}$ is the price of the consumption good defined in equation (\ref%
{aggregate}), and $w_{nt}^{j}L_{nt}^{j}$ is the nominal value-added
generated in sector $j$. This budget constraint assumes that trade is
balanced. In Section V we relax this assumption.

Using (\ref{BC}) in the utility function we can solve for the sectoral labor
allocation:%
\begin{equation}
L_{nt}^{j}=\arg \max E_{t-1}\left[ \log \left( \frac{%
\sum_{j=1}^{J}w_{nt}^{j}L_{nt}^{j}}{P_{nt}}\right) \right]
,s.t.:\sum\nolimits_{j=1}^{J}L_{nt}^{j}=L_{nt},  \label{eq:log:utility}
\end{equation}%
where $E_{t-1}$ indicates the rational expectation over
the possible realizations of period $t$ shocks. In particular, the
representative agent knows the previous values of the shock processes $%
A_{nt-1}^{j}$. The agent also knows the distribution of $A_{nt}^{j}$%
 conditional on $A_{nt-1}^{j}$ (which we estimate later
from data), and is therefore able to compute the rational expectation in (%
\ref{eq:log:utility}).\footnote{%
Implicit in our formulation is the assumption that there is perfect
risk-sharing within a country, but no risk-sharing across countries. To
motivate the lack of risk-sharing across countries, see our earlier
discussion on the high comovement between consumption and output as well as
the high correlation between consumption and output volatility.}

\subsection{Model Solution}

Conditional on the realization of the country-and-sector specific shocks $%
A_{nt}^{j}$, our model is very similar to other general equilibrium,
multi-sector versions of the Eaton-Kortum model. The main difference is that
equipped labor is pre-allocated across sectors. Hence, we do not offer a
detailed derivation of the key equilibrium conditions that are unaffected by
the ex-ante allocation of resources, but merely state them in the following
list.

\begin{equation}
d_{nmt}^{j}=\frac{T_{m}^{j}\left( \frac{B^{j}\left( w_{mt}^{j}\right)
^{\beta ^{j}}\prod_{k=1}^{J}(P_{mt}^{k})^{\gamma ^{kj}}}{A_{mt}^{j}\kappa
_{nmt}^{j}}\right) ^{-\theta }} {\sum_{i=1}^{N}T_{i}^{j}\left( \frac{%
B^{j}\left( w_{it}^{j}\right) ^{\beta
^{j}}\prod_{k=1}^{J}(P_{it}^{k})^{\gamma ^{kj}}}{A_{it}^{j}\kappa _{nit}^{j}}%
\right) ^{-\theta }},  \label{shares}
\end{equation}

\begin{equation}
P_{nt}^{j}=\xi \sum_{m=1}^{N}T_{m}^{j}\left( \frac{B^{j}\left(
w_{mt}^{j}\right) ^{\beta ^{j}}\prod_{k=1}^{J}(P_{mt}^{k})^{\gamma ^{kj}}}{%
A_{mt}^{j}\kappa _{nmt}^{j}}\right) ,  \label{eq2}
\end{equation}

\begin{equation}
P_{nt}=\prod\nolimits_{j}^{J}\left( \frac{1}{\alpha _{n}^{j}}\right)
^{\alpha ^{j}}\left( P_{nt}^{j}\right) ^{\alpha ^{j}},  \label{eq1}
\end{equation}%
\begin{equation}
R_{nt}^{j}=\sum\nolimits_{m=1}^{N}d_{mnt}^{j}E_{mt}^{j},  \label{rev}
\end{equation}%
\begin{equation}
E_{nt}^{j}=\alpha _{t}^{j}P_{nt}C_{nt}+\sum\nolimits_{k=1}^{J}\gamma
^{jk}R_{nt}^{k},  \label{exp}
\end{equation}%
\begin{equation}
w_{nt}^{j}L_{nt}^{j}=\beta ^{j}R_{nt}^{j},  \label{RC}
\end{equation}%
and the budget constraint (\ref{BC}). In the equations above, $d_{nmt}^{j}$
is the fraction of country $n$'s total spending on sector-$j$ goods that is
imported from country $m;$ $P_{nt}^{j}$ is the price of sectoral-good $j$ in
country $n$; $R_{nt}^{j}$ is total revenues accruing to firms operating in
sector $j$ in country $n$; and $E_{nt}^{j}$ is total expenditure by country $%
n$ residents (consumers and firms) on sectoral good $j$. $B^{j}\equiv \left(
\beta ^{j}\right) ^{-\beta ^{j}}\dprod\limits_{k=1}^{J}\left( \gamma
^{k_{j}}\right) ^{-\gamma ^{k_{j}}}$ and $\xi \equiv \left[ \Gamma \left( 
\frac{\theta +1-\eta }{\theta }\right) \right] $, where $\Gamma $ is the
gamma function, are parametric constants. Hence, equation (\ref{shares})
says that country $n$ imports disproportionately from countries $m$ and
sectors $j$ that have high productivity draws $T_{m}^{j}$ and $A_{mt}^{j}$;
low wages $w_{mt}^{j}$ and sectoral prices $P_{mt}^{k}$; and low bilateral
trading costs, namely high $\kappa _{nmt}$s. Equation (\ref{eq2}) says that
the same factors affect domestic sectoral prices. Equation (\ref{eq1})
follows from the final-good producer's profit maximization problem, and
shows the price of consumption as an aggregate of the sectoral prices.
Equation (\ref{rev}) expresses the total sales of sector $j$ in country $n$
as a function of each country's expenditures on that sector and the share of
country $n$ in each country's imports in that sector. Equation (\ref{exp})
states that a country's expenditures in sector $j$ is the sum of final and
intermediate uses of sector $j$ goods. Equation (\ref{RC}) simply notes from
the Cobb-Douglas formulation that value added from sector $j$ is a share $%
\beta ^{j}$ of the gross output of sector $j$.

To these fairly standard equilibrium conditions we add here the first-order
conditions for the allocation of inputs to sectors, i.e. the solution to (%
\ref{eq:log:utility}). This turns out to be: 
\begin{equation}
\frac{L_{nt}^{j}}{L_{nt}}=E_{t-1}\left[ \frac{w_{nt}^{j}L_{nt}^{j}}{%
\sum_{k}w_{nt}^{k}L_{nt}^{k}}\right] ,\qquad \forall j,t.  \label{eq:FOC}
\end{equation}%
The share of resources allocated to a given sector equals its expected share
in value added. Note that $1/\sum_{k}w_{nt}^{k}L_{nt}^{k}$ is the marginal
utility of consumption in period $t$; thus, more resources are allocated to
higher value-added sectors, after appropriately weighting by marginal
utility.\footnote{%
Compared to the allocation in a determinisitc model, in our stochastic
application sectors whose productivity is negatively correlated with
aggregate productivity (that is, they have high value added when the rest of
the economy has low value added) are allocated a disproportionate share of
resources. In states of the world in which overall income is low, the
marginal utility of consumption $1/\sum_{k}w_{nt}^{k}L_{nt}^{k}$ will be
high and hence the optimal allocation entails allocating more resources to
this sector.}

The model can conceptually be solved backwards in two steps. First, for any
given set of values for $L_{nt}^{j}$, equations (\ref{shares})-(\ref{RC})
can be solved for $P_{nt}$, $w_{nt}^{j}$, $P_{nt}^{j}$, $d_{nmt}^{j}$, $%
E_{nt}^{j}$, $R_{nt}^{j}$, and $C_{nt}$ as functions of the $\kappa
_{mnt}^{j}$s, the $T_{nt}^{j}$s, the $A_{nt}^{j}$s, and of course the $%
L_{nt}^{j}$s. For calibration purposes it turns out to be both possible and
convenient to express the dependence of these solutions on $T_{n}^{j}$, $%
A_{nt}^{j}$, and $L_{nt}^{j}$ in terms of the \textit{augmented productivity
factors}%
\begin{equation}
Z_{nt}^{j}\equiv T_{n}^{j}\left( A_{nt}^{j}\right) ^{\theta }\left(
L_{nt}\right) ^{\beta ^{j}\theta }  \label{productivityfactor}
\end{equation}%
and the sectoral employment shares $\frac{L_{nt}^{j}}{L_{nt}}$. The
augmented productivity factors capture the joint influence of all the
exogenous processes (whether deterministic or stochastic) that impinge on
the country and sector overall productive capacity.

The second stage of the solution uses (\ref{eq:FOC}) to find the
ex-ante shares $L_{nt}^{j}/L_{nt}$. Our solution method computes
the rational expectation in (\ref{eq:FOC}) by drawing from the
estimated distribution of $A_{nt}^{j}$. In particular, we begin
with a choice of candidate values for the $L_{nt}^{j}$s, and draw a
large number of realizations of the $A_{nt}^{j}$s from their
estimated distributions (conditional on the $A_{nt-1}^{j}$s). For
each of these realizations, we compute the solution for the $w_{nt}^{j}$%
s from the system (\ref{shares})-(\ref{RC}), and then the term in
brackets on the right side of (\ref{eq:FOC}). The rational expectation is
then the average of the terms in brackets across all the simulated
realizations. If this is (close enough to being) equal to the starting guess
for $L_{nt}^{j}/L_{nt}$ we stop. Otherwise, we move to a new guess
for $L_{nt}^{j}/L_{nt}$. A more detailed explanation is provided in
the Appendix.

The key theoretical outcome we are interested in is aggregate income
volatility, which we measure as the variance (or standard deviation, where
indicated), of real income deviations from country-specific trends. In turn,
real income in the model is given by total value added deflated by the
optimal expenditure-based price index, or $Y_{nt}=\frac{w_{nt}L_{nt}}{P_{nt}}%
.$ As discussed in the Introduction, these welfare-relevant measures of
income are expected to show first-order responses to changes in the terms of
trade, and hence in foreign productivities, endowments, or trade costs.%
\footnote{%
In contrast, if we were to deflate nominal GDP by using the CES price
aggregates of the sector-level variety baskets, we would retrieve the
Kehoe-Ruhl invariance of GDP to shocks to the terms of trade. It is
doubtful, however, that GDP as constructed by statistical agencies maps well
into this theoretical construct. They may measure the price of a
representative variety within each sector, the average price of an aggregate
variety basket, or a random sample of continuously used varieties. This
choice might also depend on the source country, as import price indexes are
computed differently from producer price indexes (Nakamura and Steinsson,
2008 and Nakamura and Steinsson, 2012). In contrast, the CPI is
easier to map to our model, because consumers only consume $J$ different final
goods, not a continuum of varieties. Our welfare-relevant price index, which
is the geometric average of final good prices is a very close approximation
of the expenditure-weighted T\"{o}rnqvist price index, the way the CPI is
usually calculated.}

\subsection{Two Illustrative Cases: Autarky and Costless Trade}

To illustrate our novel mechanism of diversification through trade, we begin
by analyzing a one-sector version of the model (that is, the original
Eaton-Kortum model) under two extreme cases for which we have closed-form
analytical solutions: autarky ($\kappa _{nmt}=0$ for all $n\neq m,t$) and
costless trade ($\kappa _{nmt}=1$ for all $n,m,t$). We accordingly drop the
sector subscripts. The final good is still used as an intermediate. Note
that in both cases we can set $P_{n}=1$ for all $n$. In the autarky case
this is an innocuous normalization. In the costless-trade case this is due
to the fact that prices are equalized across countries.

\subsubsection{Volatility under Autarky}

Under complete autarky, it can be easily shown that value added in the
one-sector economy is a function of augmented productivity: 
\begin{equation*}
Y_{nt}\propto \left( Z_{nt}\right) ^{\frac{1}{\beta \theta }}
\end{equation*}%
where, recall, $Z_{nt}\equiv T_{n}\left( L_{nt}A_{nt}^{1/\beta }\right)
^{\beta \theta }$. Defining $\hat{Z}_{nt}$ ($\hat{Y}_{nt}$) as the
log-deviation of $Z_{nt}$ ($Y_{nt}$) from its deterministic trend, we thus
have $\hat{Y}_{nt}=\frac{1}{\beta \theta }\hat{Z}_{nt}.$ Hence, much as in a
RBC model, in the one-sector economy under autarky shocks to value added are
driven exclusively by domestic shocks to the productive capacity of the
economy, $\hat{Z}_{nt}.$ The variance of income, $Var(\hat{Y}_{nt})$ thus
depends on the variance of the shocks $Var(\hat{Z}_{nt})$: 
\begin{equation*}
Var(\hat{Y}_{nt})=\frac{1}{\left( \beta \theta \right) ^{2}}Var(\hat{Z}%
_{nt}).
\end{equation*}

\subsubsection{Volatility under Costless International Trade}

Under costless international trade ($\kappa _{nmt}=1$) in the one-sector
economy income per capita is:\footnote{%
See derivations in the Appendix. With costless international trade, the
aggregate production function exhibits decreasing returns in the domestic
equipped labour $L_{nt}$, a result that goes back to Acemoglu and Ventura
(2002).} 
\begin{equation*}
Y_{nt}=\left( \xi B\right) ^{1/\beta }Z_{nt}^{\frac{1}{1+\beta \theta }%
}\left( \sum\nolimits_{m=1}^{N}Z_{mt}^{\frac{1}{1+\beta \theta }}\right) ^{%
\frac{1}{\beta \theta }}
\end{equation*}%
and hence income fluctuations are given by: 
\begin{equation*}
\hat{Y}_{nt}=\frac{1}{1+\beta \theta }\left[ \hat{Z}_{n}+\frac{1}{\beta
\theta }\sum\nolimits_{m=1}^{N}\gamma _{m}\hat{Z}_{m}\right] 
\end{equation*}%
where $\gamma _{m}=\frac{\bar{Z}_{m}^{\frac{1}{1+\beta \theta }}}{%
\sum_{i=1}^{N}\bar{Z}_{i}^{\frac{1}{1+\beta \theta }}}$ is the relative size
of country $j$ evaluated at the mean of $Z_{j}s$. Rearranging, we obtain $%
\hat{Y}_{nt}=\frac{1}{\beta \theta }\left[ \frac{\gamma _{n}+\beta \theta }{%
1+\beta \theta }\hat{Z}_{n}+\frac{1}{1+\beta \theta }\sum_{m\neq
n}^{N}\gamma _{m}\hat{Z}_{m}\right] $. Volatility under free trade is hence
given by: 
\begin{equation*}
Var(\hat{Y}_{nt})=\left( \frac{1}{\beta \theta }\right) ^{2}\left\{ 
\begin{array}{c}
\left( \frac{\gamma _{n}+\beta \theta }{1+\beta \theta }\right) ^{2}Var(\hat{%
Z}_{nt})+\left[ \frac{1}{1+\beta \theta }\right] ^{2}\sum_{m\neq i}\gamma
_{m}^{2}Var(\hat{Z}_{mt}) \\ 
2\frac{\gamma _{n}+\beta \theta }{1+\beta \theta }\frac{1}{1+\beta \theta }%
\sum_{m\neq n}\gamma _{m}Cov(\hat{Z}_{m,}\hat{Z}_{n})%
\end{array}%
\right\} 
\end{equation*}

Compared to the variance in autarky, $\frac{1}{\left( \beta \theta \right)
^{2}}V(\hat{Z}_{nt})$, it is clear that the volatility due to domestic
productivity fluctuations, $Var(\hat{Z}_{nt}),$ now receives a smaller
loading, as $\left( \frac{\gamma _{n}+\beta \theta }{1+\beta \theta }\right)
^{2}<1$ since $\gamma _{n}<1.$ The smaller the country (as gauged by its
share $\gamma _{n}$), the smaller the impact of domestic volatility of
shocks, $\hat{Z}_{n},$ on its income, when compared to autarky. Openness to
trade, however, exposes the economy to other countries' productivity shocks,
which will also contribute to the country's overall volatility.

Whether or not the gain in diversification (given by lower exposure to
domestic productivity) is bigger than the increased exposure to new shocks
depends on the variance-covariance matrix of shocks across countries. If all
countries have the same constant variance $Var(\hat{Z}_{nt})=\sigma ,$ and
the $\hat{Z}_{nt}$ are uncorrelated, volatility under free trade becomes: 
\begin{equation*}
Var(\hat{Y}_{nt})=\left( \frac{1}{\beta \theta }\right) ^{2}\left\{ \left( 
\frac{\gamma _{n}+\beta \theta }{1+\beta \theta }\right) ^{2}+\left[ \frac{1%
}{1+\beta \theta }\right] ^{2}\sum_{m\neq i}\gamma _{m}^{2}\right\} \sigma
\end{equation*}%
which is unambiguously lower than the volatility under autarky.\footnote{%
To see this note that $2\beta \theta \gamma _{n}+\sum_{j=1}\gamma
_{j}^{2}<2\beta \theta +1$ since $\gamma _{m}\leq 1$ for every $m$, and so $%
\left( \beta \theta \right) ^{2}+2\beta \theta \gamma _{n}+\sum_{j=1}\gamma
_{j}^{2}<(1+\beta \theta )^{2}$. This means that the expression in curly
brackets is less than 1.} Of course, if other countries have higher
variances or the covariance terms are important, then the weights countries
receive matter and the resulting change in volatility cannot be
unambiguously signed.

Aside from the over-simplified variance and covariance structure, these
examples abstract from the traditional channel thought to link trade to
increased volatility, namely sectoral specialization. In order to evaluate
the relative importance of country diversification and sectoral
specialization, as well as to base the analysis on a more realistic
stochastic environment based on the data, and to evaluate infra-marginal
changes in trade costs, the rest of the paper focuses on the full
multi-sector model with frictions to the reallocation of labor following the
realization of shocks.

\section{Quantification}

\label{mapping_into_observables} Our goal is to quantitatively assess the
effect of historical changes in trade barriers on income volatility for as
large a sample of countries and as fine a level of sectoral disaggregation
as available data allows. It turns out that the necessary data are available
for a sample of 24\ core countries, and an aggregate of the remaining
countries,\ to which we refer to as \textquotedblleft rest of the
world\textquotedblright\ (ROW). The country coverage is good, in the sense
that the countries included account for an overwhelming share of world GDP
and trade. In terms of sectoral breakdown, we are able to consider 24
sectors: agriculture, 22 manufacturing sectors, and services. It would
clearly have been desirable to access an even finer breakdown. Among other
things, a finer breakdown would have potentially implied greater effective
rigidity in the allocation of labor across sectors, allowing us to test the
robustness of our conclusions on the importance of the specialization
channel. Nevertheless, 24 sectors is at the top end of the level of
disaggregation usually achieved in applications of the Eaton-Kortum
framework.

In order to solve the model\ numerically, we need to estimate the values of
the exogenous trading costs $\kappa _{nmt}^{j}$ and the augmented
productivity processes $Z_{nt}^{j}$. We also need to calibrate the
parameters $\alpha _{t}^{j}$, $\beta ^{j}$, $\gamma ^{kj}$, $\theta $, and $%
\eta $.

\subsection{Exogenous Processes}

As has become standard in empirical applications of the Eaton and Kortum
framework, we back out realized paths of both trade costs $\kappa _{nmt}^{j}$
and augmented productivities $Z_{nt}^{j}$ from (versions of) the gravity
equation (\ref{shares}) [e.g. Costinot, Donaldson, Komunjer (2012),
Levchenko and Zhang (2014, 2016)]. Allen, Arkolakis, and Takahashi (2017)
discuss the identification issues involved in this inference problem, whose
solution generally requires additional information on trade costs. In our
case, we impose additional restrictions on the patterns of bilateral trade
costs, which allow us to back out the full matrix of bilateral trade costs $%
\kappa _{nmt}^{j}$ independently from the $Z_{nt}^{j}$s. We can then plug
the estimated $\kappa _{nmt}^{j}$s back into (\ref{shares}) to back out the $%
Z_{nt}^{j}$s.\footnote{%
An alternative to our two-step strategy is to find proxies for the
observable determinants of trade costs (e.g. distance, or colonial links)
and model the $\kappa $s explicitly as functions of these determinants. Then
equation (\ref{shares}) can be estimated econometrically and the $Z$s
recoverd as (functions of) country-sector fixed effects. See, e.g.,
Levchenko and Zhang (2014).}

\subsubsection{Trade Costs}

In order to back out the $\kappa _{nmt}^{j}$s independently of the other
variables in the gravity equation we follow Head and Ries (2001) and assume
that $\kappa _{nmt}^{j}=1$ for $n=m$, and that $\kappa _{nmt}^{j}=\kappa
_{mnt}^{j}$ for all $n$, $m$, and $j$. With these assumptions, equation (\ref%
{shares}) can be manipulated to yield:%
\begin{equation}
\frac{d_{nmt}^{j}d_{mnt}^{j}}{d_{mmt}^{j}d_{nnt}^{j}}=\left( \kappa
_{nmt}^{j}\right) ^{2\theta }.  \label{kappa}
\end{equation}%
Recall that $d_{nmt}^{j}$ is the fraction of country $n$'s total spending on
sector-$j$ goods that is imported from country $m$. Imports are directly
observable and spending can be constructed from available data as gross
sectoral output plus sectoral imports minus sectoral exports. Hence, for a
given value of $\theta $ (see below for the calibration of this parameter),
we can obtain the time series of trading costs by sector and country-pairs $%
\left\{ \kappa _{nmt}^{j}\right\} $.

Figure 1 shows the histograms of bilateral $\kappa $s in manufacturing and
agriculture in the first and last year of our sample (recall that services
are treated as a nontradable sector). In both agriculture and manufacturing
trade barriers have declined significantly since the early 1970s. As is
typical of estimated trade costs from gravity equations the \textit{levels}
of the trade costs are very large. But it is important to remember that the
trade barriers do not only reflect transport costs and tariff and non-tariff
trade barriers; but also that many manufacturing and, especially,
agricultural goods are not fully tradable (e.g. perishable products). They
may also pick up a home-bias effect that is not explicitly modelled in Eaton
and Kortum.

\subsubsection{\label{PTS}Productivity in Tradable Sectors}

Using again (\ref{shares}), together with (\ref{eq2}) and our definition of
augmented productivity (\ref{productivityfactor}), some algebra yields 
\begin{equation}
Z_{nt}^{j}=\underbrace{{B^{j}}^{\theta }{\xi }^{\theta }d_{mnt}^{j}\left(
y_{n}^{j}\right) ^{\theta \beta ^{j}}\left( \kappa _{mnt}^{j}\right)
^{-\theta }\left( P_{nt}^{\beta
^{j}}\prod\nolimits_{k=1}^{J}(P_{nt}^{k})^{\gamma ^{kj}}\right) ^{\theta
}\left( \psi _{nt}^{j}\right) ^{-\theta \beta ^{j}}}_{\equiv \exp (\zeta
_{mnt}^{j})}\left( {P_{mt}^{j}}\right) ^{-\theta },  \label{Z}
\end{equation}%
where $\psi _{n}^{j}\equiv \frac{L_{n}^{j}}{L_{n}}$ and $y_{n}^{j}\equiv 
\frac{L_{n}^{j}w_{n}^{j}}{P_{n}}$. This equation holds for all $n,m,j,t$. It
says that, for a given price of sectoral good $j$ in country $m$, ${%
P_{mt}^{j}}$, and bilateral trading costs $\kappa _{mnt}^{j}$, productivity
in country $n$ in that sector is inferred to be high if country $n$ exports
a lot to country $m$, or $d_{mnt}^{j}$ is large; if value added $y_{n}^{j}$
is large; or the sectoral input share $\psi _{n}^{j}$ is large.

For all countries, we can directly observe several of the terms collected in
the object we have called $\exp (\zeta _{mnt}^{j})$. In particular, data is
available for sectoral import shares $d_{mnt}^{j}$ (as already used in the
previous subsection), sectoral value added $y_{nt}^{j}$, and aggregate
prices $P_{nt}$. We do not observe directly the sectoral shares $\psi
_{nt}^{j}$. However, recall from equation (\ref{eq:FOC}) that, in our model,
equipped labour is allocated across sectors so that the share of employment
in each sector equals the \textit{expected} share of that sector's value
added in total value added. Hence, to compute the terms $\psi _{nt}^{j}$ we
take the time series of sector $j$'s value added in country $n$'s value
added, and extract its (nonlinear) time trend. We then treat the trend as a
proxy for the expected value and plug it into (\ref{Z}) as our estimate of $%
\psi _{nt}^{j}.$\footnote{%
An alternative procedure would be to take a stand on the equipped-labor
aggregate. For example, Levchenko and Zhang (2014) assume it is a
Cobb-Douglas aggregate of capital and (raw) labor.}

This leaves us needing the sector-specific price deflators ${P_{mt}^{j}}$
for some benchmark country $m$. We could easily just plug into (\ref{Z}) the
US sectoral price indices and use them to recover the $Z_{nt}^{j}$s for all
other countries (and the US itself). It turns out, however, that in the next
subsection we will need sectoral price deflators for tradable sectors for
all countries in order to obtain estimates of the productivity processes for
the nontradable sector. As these sectoral price indices are not available
for many of the countries in our sample, we develop here a procedure to back
out tradeable prices. When we have tradable prices for all countries, we can
use (\ref{Z}) more efficiently to estimate productivity processes.

Taking logs and rearranging (\ref{Z}) yields.%
\begin{equation*}
\theta \log \left( P_{mt}^{j}\right) =\zeta _{mnt}^{j}-\log \left(
Z_{nt}^{j}\right) .
\end{equation*}%
Since this relationship (vis-a-vis) country $n$ must hold for any generic
countries $m$ and $m^{\prime }$, we can write%
\begin{equation*}
\theta \log \left( P_{mt}^{j}\right) -\theta \log \left( P_{m^{\prime
}t}^{j}\right) =\zeta _{mnt}^{j}-\zeta _{m^{\prime }nt}^{j}.
\end{equation*}%
Rearranging this, and averaging over $n$, we further get%
\begin{equation*}
\theta \log \left( P_{mt}^{j}\right) =\frac{1}{N}\sum\nolimits_{n=1}^{N}%
\left( \zeta _{mnt}^{j}-\zeta _{m^{\prime }nt}^{j}\right) +\theta \log
\left( P_{m^{\prime }t}^{j}\right) .
\end{equation*}%
Recalling that the $\zeta $s are observable for all $n$, this expression
tells us that we can recover the sectoral prices for any country $m$ if we
have sectoral price indices for at least one country $m^{\prime }$. We do
have sectoral price indices for the US. We choose units of accounts for each
sector so that U.S. nominal sectoral prices are equal to 1 in 1972.

Having thus obtained sectoral price series $P_{mt}^{j}$ for all countries
and sectors, we can return to (\ref{Z}) and recover $Z_{nt}^{j}$ from%
\begin{equation*}
\log (Z_{nt}^{j})=\frac{1}{N}\sum\nolimits_{m=1}^{N}\left[ \zeta
_{mnt}^{j}-\theta \log \left( P_{mt}^{j}\right) \right] .
\end{equation*}%
Note that, in the last two expressions, instead of using the average across
a country's trade partners we could have used any individual bilateral
relation. Theoretically, either option is valid. However, using the average
minimizes the influence of measurement error.

\subsubsection{\label{PNTS}Productivity in Nontradables}

The procedure in the previous subsection uses data on trade flows and is
thus only applicable to the recovery of augmented productivities in the
tradable sectors: agriculture and the various manufacturing industries. To
recover the productivity series in the service sector we begin by
constructing a time series for the price of services. From equation (\ref%
{eq1}), the price of services $P_{n,t}^{s}$ can be written as%
\begin{equation*}
P_{nt}^{s}=\left( \frac{P_{nt}}{P_{US,t}}P_{US,t}\right) ^{\frac{1}{\alpha
^{s}}}\left( \prod\nolimits_{j=1}^{J}{\alpha ^{j}}^{-\alpha ^{j}}\right) ^{-%
\frac{1}{\alpha ^{s}}}\left[ \prod\nolimits_{j\neq s}\left(
P_{nt}^{j}\right) ^{\alpha ^{j}}\right] ^{-\frac{1}{\alpha ^{s}}}.
\end{equation*}%
We have just described in the previous subsection how to estimate the prices
of all the sectors other than services, i.e. the $P_{nt}^{j}$s in the last
term. From the Penn World Tables we can obtain a general price index for
each country $n$ relative to the United States, $\frac{P_{nt}}{P_{US,t}}$.
And $P_{US,t}$ is simply the US general price index. With the price series
for services at hand, we can construct augmented productivity in services, $%
Z_{nt}^{s}$ using again equation (\ref{Z}), for the case $n=m$.\footnote{%
Implying, therefore, $d_{mnt}^{s}=\kappa_{mnt}^s=1$.}

\subsubsection{\label{SvAS}Sectoral versus Aggregate Shocks}

Since we are interested in decomposing the overall effect of trade on
volatility into the contributions of the two mechanisms, specialization and
diversification, we need to separately identify sectoral and aggregate
shocks. We resort to a factor model that decomposes augmented productivity
shocks into sector- and country-specific components, as described in Koren
and Tenreyro (2007). To separate per period shocks from trends we use a
band-pass filter to detrend each $\left\{ \log {Z_{nt}^{j}}\right\}
_{t=1}^{T}$ series. Without loss of generality, we decompose the cyclical
component, denoted $\hat{Z}_{nt}^{j}$, as: 
\begin{equation*}
\hat{Z}_{nt}^{j}=\lambda _{t}^{j}+\mu _{nt}+\epsilon _{nt}^{j},
\end{equation*}%
where ${\mu _{nt}}$ is the country-specific factor, affecting all sectors
within the country; $\lambda _{t}^{j}$ is the global sectoral factor,
affecting sector $j$ in all countries;\ and the residual $\epsilon _{nt}^{j}$
is the idiosyncratic component, specific to the country and sector.\footnote{%
The three factors, $\lambda ,\mu $, and $\epsilon $ are estimated as: 
\begin{align*}
\lambda _{t}^{j}& =N^{-1}\sum_{n=1}^{N}\hat{Z}_{nt}^{j} \\
\mu _{nt}& =J^{-1}\sum_{j=1}^{J}\bar{\alpha}^{j}\left( \hat{Z}%
_{nt}^{j}-\lambda _{t}^{j}\right) \\
\epsilon _{nt}^{j}& =\hat{Z}_{nt}^{j}-\lambda _{t}^{j}-\mu _{nt}\text{,}
\end{align*}%
where $\bar{\alpha}^{j}$ is the time average of sectoral expenditure shares $%
\alpha _{t}^{j}$, and we impose the restriction $\sum_{n}{\mu _{n}}=0,$
implying that the country-specific effect is expressed relative to the
world's aggregate. We calculate the country factor as a weighted average of
shocks, because the single sector of services takes up 70-80 percent of
value added in many economies. This is in contrast to Koren and Tenreyro
(2007), who use unweighted average. Their application focuses on
manufacturing sectors, which do not differ as much in size.} In the
counterfactual exercises, we can mute the sector- or country-specific
factors by setting the corresponding components equal to 0, in order to
identify the separate effects of the two trade channels affecting volatility.

In order to solve the model, we must be able to compute the rational
expectation of a sector's value added as a share of total value added at the
beginning of period $t$, conditional on $\hat{Z}_{nt-1}^{j}$%
 (see equation (\ref{eq:FOC})). To do this we assume that the
processes $\lambda _{t}^{j}$, $\mu _{nt}$, and $\epsilon
_{nt}^{j}$ are all AR(1) log-normal processes, and for each of them
we estimate specific autoregressive and variance parameters. We then
construct possible realizations of $Z_{nt}^{j}$ by drawing
from the estimated distributions of $\lambda _{t}^{j}$, $\mu _{nt}$,
and $\epsilon _{nt}^{j}$ conditional on $\lambda _{t-1}^{j}$%
, $\mu _{nt-1}$, and $\epsilon _{nt-1}^{j}$, and
adding the drawn values to the deterministic (trend) component of log $%
Z_{nt}^{j}$.

\subsection{Calibration}

We set $\alpha _{t}^{j}$ so as to match the cross-country average of the
share of sector $j$ in total final uses, in each year, using the data on
value added described in the Appendix. The $\beta ^{j}$s are calculated as
the average ratios (across time and countries) of value added to total
output in each sector, again using the sectoral value added and gross output
data from the appendix. And the $\gamma ^{kj}$s are the average shares of
purchases by sector $j$ from sector $k$ from the OECD input-output tables,
as a share of total sectoral output.

We allow for a relatively broad parametric range for $\theta $, from $\theta
=2$ to $\theta =8,$ consistent with the estimates in the literature (see
Eaton and Kortum, 2002, Donaldson 2015, and Simonovska and Waugh, 2014). We
use $\theta =4$ as the baseline case, and report the results for other
values when discussing the sensitivity of our results. We calibrate the
elasticity of substitution across varieties $\eta =4$, consistent with Broda
and Weinstein (2006)'s median estimates. The results are not sensitive to
this parametric choice.

\section{The Effect of Trade on Volatility}

This section uses the framework developed above to quantitatively assess how
historical changes in trade costs from the early 1970s have affected
volatility patterns in a sample of countries at different levels of
development. We first analyze the baseline model's results and then perform
a series of sensitivity checks and extensions.

\subsection{Baseline Results}

Figure 2 starts by comparing the baseline model-generated income volatility
with the volatility in the data. The baseline model uses our benchmark
calibration, $\theta =4$, and feeds in the historical time series for the
trade costs $\kappa _{mnt}$, and for the augmented productivity factors $%
Z_{nt}^{j}$. The graph shows the standard deviation of real income
deviations from trend. Recall that real income is measured as value added
deflated by the expenditure-based price index. The data counterpart is
nominal GDP deflated by the CPI index. The correlation between standard
deviations in the model and data series is 0.96 (0.88 without China) for the
standard deviation and 0.99 (0.89 without China) for the variance. The
analysis that follows will focus on the variance as a measure of volatility,
rather than the standard deviation, because we exploit the additivity
properties of the former to separately account for the diversification and
sectoral-specialization effects.

Table 1 investigates how the changes in trading costs have affected
volatility in the 24 countries in our sample (plus the rest of the world).
Column 1 compares our baseline scenario, which uses the estimated time paths
of trading costs and productivity processes, to a scenario in which we
remove the secular decline in trading costs.\footnote{%
The absolute numbers of the volatilities generated by the scenarios
discussed in this Section are reported in Appendix Table 1.} In particular,
in the counterfactual scenario we keep all the $\kappa _{nmt}^{j}$s constant
at their 1972 level. The column shows volatility under the counterfactual
minus volatility in the baseline, and this difference taken as a percentage
of the volatility at constant trading costs. The numbers can be interpreted
as the proportional change in volatility caused by the decline in trading
costs.

The comparison in Column 1 reveals that volatility is generally higher under
the counterfactual scenario with constant trading costs than in the
baseline. For all countries except for China, there would have been more
volatility under constant trade costs than there has been. For almost all
countries, therefore, the common wisdom which predicts greater volatility
following trade integration does not seem to apply.

The biggest declines in volatility caused by trade occurred in
Belgium-Luxemburg, Canada, Denmark, Germany, Ireland, Mexico, the
Netherlands, Spain, and the United Kingdom, all of which saw volatility
reductions due to trade in excess of 50\% (meaning their volatility has been
50 percent lower than it would have been had trading costs stayed at their
1972 levels). In the one country in which trade has created additional
volatility, the excess volatility is negligible. The (unweighted) average
country in our sample experienced a 36\% decline in volatility thanks to
increased openness. But this average effect masks a huge amount of
heterogeneity in the quantitative and qualitative effect of trade in
volatility, consistent with our discussion of the country-specificity of the
trade-volatility relation.

As discussed at several points, openness affects volatility through two
channels: a diversification effect and a specialization effect. While
neither effect has an unambiguous impact, it is sensible to expect the
diversification effect to reduce the impact of country-specific shocks, and
hence - in most cases - to reduce volatility; similarly, by exacerbating the
impact of sectoral shocks, the specialization effect is generally deemed to
increase volatility. In the rest of the table we assess and quantify these
predictions.

In order to quantify the impact of the diversification effect, we compare
two counter-factual scenarios. As before, the two scenarios differ in the
path of trading costs, with one scenario featuring the same decline in
trading cost that we back out from the data, and the other having trading
costs constant at 1972 levels. However, in these two scenarios the series
for $Z_{nt}^{j}$ is replaced by a modified series from which we remove all
sectoral shocks (i.e. the shocks $\lambda _{t}^{j}$ and $\varepsilon
_{t}^{j} $ defined in Section \ref{SvAS}). In other words we ask what
volatility would have been with and without the observed decline in trade
costs, if the only shocks to productivity had been the country-wide shocks.
Because these two scenarios do not feature sectoral shocks, any differences
in volatility must be ascribed to the diversification effect.

The difference is again expressed as a percentage of the volatility under
the 1972's trading cost levels\ and is reported in Column 2. Once again,
overwhelmingly volatility at 1972 trade barriers is larger than volatility
in the baseline case, confirming that the diversification channel strongly
operates in the direction of lower volatility - as expected. It is
interesting though that there are a few countries for which volatility is
lower at 1972 trade costs. As discussed, even the diversification channel
can amplify volatility, if openness exposes a country to disproportionately
large and volatile trading partners, or partners whose shocks are highly
correlated with a country's own. Evidently this was the case for these
countries. On average, the diversification channel induces a 41\% drop in
volatility relative to the case where barriers are held at the initial value.

Because of the additive properties of the variance, the specialization
effect can be quantified as the difference between the overall change in
volatility, and the change due to the diversification effect. This is
reported in Column 3. The figures should be interpreted as the change in
volatility due to trade integration that would have occurred if only
sectoral shocks (global or country specific) had been present. The change is
positive for 15 out of 24 countries. Consistent with the standard view,
therefore, the specialization channel tends to increase volatility in a
majority of cases. Remarkably, however, there is a large number of countries
which evidently are pushed to specialize into less volatile sectors, or into
sectors that comove negatively (or less positively) with the country's
aggregate shocks or other sectoral shocks. On average, the specialization
channel implies an increase in volatility of 5\%.

The most important lesson from the comparison of Columns 2 and 3 is about
the relative magnitude of the diversification and specialization effects.
The average change due to the diversification mechanism is about eight times
as large as the average change due to the specialization mechanism. The
specialization effect, on which the policy debate seems centred, is not as
important as the diversification effect. We have hinted at the likely reason
for this in the Introduction: country-specific shocks are simply much more
important quantitatively than sector-specific ones.

In Table 2 we briefly present a dynamic view of how the overall changes seen
in Table 1 came about. As Table 1, the table presents comparisons of
volatility under different scenarios, but volatility is computed by decade.%
\footnote{%
To calculate decadal volatility, we compute the variance of annual log
growth rates in real GDP. It is infeasible to estimate a band-pass filter
given just 10 years of data. The overall magnitudes of volatility are very
similar to those in Table 1.} Not surprisingly, the impact of trade
(understood as the change in trading costs since 1972) on volatility is
modest in the 1970s, as by the end of the 1970s trade costs had not had much
time to drift away from the 1972 values. Throughout the rest of the period,
the gap between actual volatility and volatility at 1972 trade costs opens
steadily, as the world economy becomes more and more integrated.

\textbf{This overall monotonic decline in volatilty, however, masks some
more nuanced dynamics of the diversification and specialization effects. In
particular, the diversification effect peters out in the period 2000-2007.
This petering out in the last seven years of the sample may possibly reflect
some noisiness due to the relative short time span over which volatilities
are computed. However, taken at face value, it points to the fact that --
consistent with our theory -- the impact of trade on volatility is not only
heterogenous across countries, but also over time. For example, the decline
in the diversification effect could be due to country-wide shocks becaming
more correlated in the 2000s.}

\subsection{Sensitivity Analysis}

In this section we evaluate the robustness of our baseline results to three
alternative implementation choices: (i) allowing for unbalanced trade; (ii)
alternative calibration values; (iii) allowing for costly labor reallocation
across sectors, \textbf{and (iv) allowing for elasticities of substitution
in consumption other than 1.}

\subsubsection{Trade Imbalances}

Our benchmark model focuses on the balanced trade case. Because we observe
significant trade imbalances during the sample period, we begin our
robustness checks by allowing countries to run trade surpluses and deficits.
We do not attempt to endogenize trade deficits as the computational
challenges of adding intertemporal considerations (including issues of
default) are formidable. Furthermore, available theoretical models of
intertemporal trade are not particularly successful empirically. Hence, as
is customary in quantitative applications of the Eaton and Kortum model, we
treat the trade surplus an as exogenous process which we take from the data.
The required modifications to the baseline model are described in the
appendix. As shown in Table 3, the quantitative results with trade
imbalances are extremely similar to those in the baseline.

\subsubsection{Scope for Comparative Advantage $\protect\theta $}

Table 4 shows the change in volatility due to international trade and its
decomposition for two other\ (extreme) values of $\theta $, $\theta =2$ and $%
\theta =8.$ The general message is qualitatively robust: i) the effect of
trade on volatility varies across countries; ii) the diversification channel
tends to reduce volatility; iii) sectoral specialization tends to reduce
volatility; (iv) the diversification channel is much more important than the
specialization channel. Having said that, the magnitude of the effects is
quite sensitive to changes in $\theta $, with the effect of trade on
volatility being stronger for lower values of $\theta $, i.e. when the scope
for comparative advantage increases.\footnote{%
This exercise underscores the importance of the parameter $\theta $, and
adds to the message of Arkolakis, Costinot, and Rodriguez-Clare (2012): in
order to assess the effects of trade on key aggregate variables, the
elasticity of trade to trade costs plays a key role.}

\subsubsection{Adjustment Costs and Ex Post Sectoral Reallocation}

The baseline model assumes that the sectoral allocation of equipped labour
is decided one period in advance, before productivity shocks are realized.
In this section we relax this stark assumption. We assume that the ex post
reallocation of equipped labour is possible, but an adjustment cost is paid
in that reallocation. By making sectoral reallocation of labor more flexible
we necessarily reduce the importance of the sectoral specialization effect,
and magnify the relative importance of our novel diversification mechanism.

We model the cost of labor reallocation in reduced-form fashion. In
particular, lifetime utility is given by 
\begin{equation}
U_{n}=\sum\limits_{t=0}^{\infty }\delta ^{t}\left\{ \log (C_{nt})-\frac{%
\varrho }{2}\sum_{j=1}^{J}\left[ \psi _{nt_{+}}^{j}-\psi _{nt_{-}}^{j}\right]
^{2}\right\} ,  \label{AC}
\end{equation}%
where $\psi _{nt_{-}}^{j}=\frac{L_{nt_{_{-}}}^{j}}{L_{nt}}$ and $\psi
_{nt_{+}}^{j}=\frac{L_{nt_{+}}^{j}}{L_{nt}}$, and $L_{nt_{_{-}}}^{j}$($%
L_{nt_{+}}^{j}$) is the equipped labour assigned to sector $j$ before
(after) observing the realization of the shocks. A higher value of $\varrho $
implies higher adjustment costs.

The ex-post sectoral input allocation solves:%
\begin{equation*}
L_{nt_{+}}^{k}=\arg \max \left[ \log \left( \frac{%
\sum_{j=1}^{J}w_{nt}^{j}L_{nt_{+}}^{j}}{P_{nt}}\right) -\frac{\varrho }{2}%
\sum_{j=1}^{J}\left[ \psi _{nt_{+}}^{j}-\psi _{nt_{-}}^{j}\right] ^{2}\right]
,\text{ }s.t.:\sum_{j=1}^{J}\psi _{nt_{+}}^{j}=\sum_{j=1}^{J}\psi
_{nt_{-}}^{j}=1,
\end{equation*}%
and the first-order conditions lead to:%
\begin{equation}
\psi _{nt_{+}}^{k}=\psi _{nt_{-}}^{k}+\frac{1}{\varrho }\left[ \frac{%
w_{nt}^{k}-\frac{1}{J}\sum_{j=1}^{J}w_{nt}^{j}}{%
\sum_{j=1}^{J}w_{nt}^{j}L_{nt_{+}}^{j}/L_{nt}}\right] .  \label{estimate}
\end{equation}%
The ex post input shares $\psi _{nt_{+}}^{k}$ equal the ex-ante optimal
shares $\psi _{nt_{-}}^{k}$ plus a fraction of the percentage differential
between the sectoral input cost $w_{nt}^{k}$ and the average equipped labour
cost in the economy $\frac{1}{J}\sum_{j=1}^{J}w_{nt}^{j}$. (Note that the
denominator is the average input cost in the economy.) The adjustment cost
parameter $\varrho $ determines the semi-elasticity of sectoral adjustment
to the cost differential.

Using (\ref{estimate}) in (\ref{AC}) we can solve for the ex-ante
allocation. The first-order condition for $\psi _{nt_{-}}^{j}$ is given by
equation (\ref{eq:FOC}). This is the same condition as in our baseline case. 
That is ex-ante labor shares should equal expected wage bill shares. 
Note, however, that the stochastic process of $w^j_{nt}$ is different with labor adjustment, so the solution to the ex-ante labor allocation problem will be different than in our baseline case.

To calibrate $\varrho $, we use EU KLEMS data on employment and compensation
for all countries in the European Union from 1970 to 2007. Using these data,
we compute the object in the square bracket in equation (\ref{estimate}). We
then regress yearly changes in labour shares on yearly changes in the wage
differentials to obtain estimates of $\frac{1}{\varrho }$. The estimated
regression coefficient is 0.001 (p-value 0.03), implying that labor
reallocation is quite unresponsive to wage differentials.\footnote{%
This result is remniniscent of Wacziarg and Wallack (2004), who find small
intersectoral labor movements in response to trade liberalizations.}

We solve the model and counterfactuals under $\frac{1}{\varrho }=0.001$ and
report the results in Table $5.$ Given the large estimated value of $\varrho 
$, the results are very similar to those in the baseline model. We have
experimented with a range of values of $\frac{1}{\varrho }$ (from 0.0005 to
0.002) and the results are virtually identical.

\subsubsection{Preferences with Constant Elasticity of Substitution}\label{Sces}

In our baseline model preferences over sectoral goods aggregate in
Cobb-Douglas fashion. In this robustness check we replace equation (\ref%
{aggregate}) by a CES formulation. This requires calibrating $J$ \
\textquotedblleft share\textquotedblright\ parameters, as well as an
elasticity of substitution. Our strategy is to calibrate the share
parameters by matching the share of each final good sector in global expenditure for each year. 
We then look at how our results vary with different
values of the elasticity of substitution.

The results are presented in Table 6. The overall effect of trade on
volatility is quite similar across different specifications of preferences,
though when substitutability is quite low (elasticity of substitution equal
to 0.5) the diversification effect and the sectoral specialization effect
change in offsetting directions

\subsection{Additional Insights from the Calibrated Model}

In this section we use our model to investigate two further questions about
the forces at work in our model and in the data. In particular we ask: (i)
What is the quantitative role of intersectoral input-output linkages in the
relationship between trade openness and volatility? And (ii) Did the
emergence of China as a global trading powerhouse exert a disproportionate
effect in other countries' volatility through trade?

\subsubsection{Input-Output Linkages}

Our model features input-output linkages as each sector produces goods that
can be used as intermediates for other sectors. It is interesting to
evaluate the role of these input-output linkages in producing our
quantitative results. In principle, we would expect the existence of
input-output linkages to provide diversification benefits to sectors, as
implicit in such linkages there are possibilities for substitution away from
inputs experiencing adverse shocks [e.g., Koren and Tenreyro (2013)].
However, similar to our discussion of the country diversification channel,
input-output linkages can also create excessive exposure to particularly
volatile suppliers, potentially leading to greater volatility relative to a
benchmark where each sector only uses non-produced inputs (or intermediates
originating from within the sector). Either way, increased openness to trade
should magnify these effects. For example, the more a country can freely
trade, the greater the opportunities for a firm to diversify among its input
suppliers, and the greater the diversification benefits associates with
input-output linkages.

To see if input-output linkages do indeed amplify the impact of trade on
income volatility in our model, we compare our baseline results to those of
an alternative model without intermediates, i.e. where we set $\gamma
^{kj}=0 $ for all $j$ and $k$ (and consequently $\beta =1$). We then
re-calibrate the productivity shocks and the trade cost processes to fit
value-added and trade data, as before. The results from this no-input-output
model are presented in Table $7$, and should as usual be compared to those
of Table 1. While the qualitative findings are similar to those of the full
model with input-output linkages, the quantitative impact of trade is
considerably reduced in their absence. The average decline in volatility due
to trade is only 3.5\% (as usual entirely by the diversification effect).
Hence, allowing firms to source inputs from other sectors is crucial to
capture the full effects of trade on volatility.

\subsubsection{The Role of China}

Our model can be used to generate additional counterfactuals that can shed
further light on the sources of changes in income volatility over the last
few decades. The emergence of China as a major global trading nation has
certainly had a significant effect on the overall openness of other
countries. Other authors have already offered evaluations of the impact of
China on the first moment of income, i.e. via the classic gains from trade
[di Giovanni, Levchenko, and Zhang (2014); Hsieh and Ossa (2016) ]; its
impact on local labor markets [Autor, Dorn, and Hansen (2013), Caliendo,
Dvorkin, and Parro (2017)]; and its influence on innovation [Bloom, Daca,
and Van Reenen (2016)]. Given China's distinct patterns of comparative
advantage and unique cyclical characteristics, it is also interesting to
assess its effects on other countries' income volatility.

We assess the role of China with two distinct thought experiments. In the
first experiment we imagine a counter-factual world where China does not
exists. That is, we perform our usual set of simulations but we drop China
from the set of countries. The changes in volatilities we report are
therefore the changes in volatility that lower trade costs among the
remaining countries would have generated if China had not been participating
in world trade. In the second experiment, we imagine a scenario in which
China does participate in world trade, but its trading costs are held
constant at 1972 levels. The changes in volatility we report are therefore
the changes in volatility that lower trade costs among the remaining
countries would have generated if China had not experienced any decline in
trade costs.

The results from these experiments are presented in Table 8. With only a few
exceptions, the impact of trade on volatility without China or when China's
trading costs are held constant at 1972 levels are broadly of a similar
magnitude. This is not too surprising as China was obviously quite closed in
1972, so holding its trade costs constant limits China's impact on other
countries in a similar way as not having China at all.

The most interesting comparison, however, is not between the two scenarios
in Table 8, but between the scenarios in Table 8 and our baseline Table 1.
The main thing to notice is that the figures in Table 1 are generally quite
close to the figures in Table 8. This means that the decline in volatility
when all countries experience trade cost declines is quite similar to the
decline in volatility when all countries bar China experience trade cost
declines, or even when China does not participate in world trade at all. Put
crudely, China does not drive our main results.

\section{Conclusions}

How does openness to trade affect income volatility? Our study challenges
the standard view that trade increases volatility. It highlights a new
mechanism (country diversification) whereby trade can lower volatility. It
also shows that the standard mechanism of sectoral specialization---usually
deemed to increase volatility---can often in practice lead to lower
volatility. The analysis indicates that diversification of country-specific
shocks has generally led to lower volatility during the period we analyze,
and has been quantitatively much more important than the specialization
mechanism. The sizeable heterogeneity in the trade effects on volatility can
contribute to understand the heterogeneity of results documented by the
existing empirical literature.

\begin{thebibliography}{99}
\bibitem{} Acemoglu, D. and J. Ventura (2002), \textquotedblleft The World
Income Distribution,\textquotedblright\ Quarterly Journal of Economics, 117
(2), p. 659-694

\bibitem{} Allen, A., C. Arkolakis, and Y. Takahashi (2017)
\textquotedblleft Universal Gravity,\textquotedblright\ Yale manuscript.

\bibitem{} Alvarez, F. and R. E. Lucas (2007), \textquotedblleft General
Equilibrium Analysis of the Eaton-Kortum Model of International
Trade,\textquotedblright\ Journal of Monetary Economics, 54 (6): 1726-1768.

\bibitem{} Anderson, J., 2011. \textquotedblleft The specific factors
continuum model, with implications for globalization and income
risk,\textquotedblright\ Journal of International Economics, Elsevier, vol.
85(2): 174-185.

\bibitem{} Arkolakis, C., A. Costinot and A. Rodriguez-Clare (2012),
\textquotedblleft New Trade Models, Same Old Gains?\textquotedblright\
American Economic Review, 2012, 102(1), p. 94-130.

\bibitem{} Arkolakis, C. and A. Ramanarayanan (2009), \textquotedblleft
Vertical Specialization and International Business Cycle
Synchronization,\textquotedblright\ Scandinavian Journal of Economics,
111(4), 655-80.

\bibitem{} Autor, David H., David Dorn, and Gordon H. Hanson. 2013. "The
China Syndrome: Local Labor Market Effects of Import Competition in the
United States." American Economic Review, 103(6): 2121-68.

\bibitem{} Backus, D., Patrick J. Kehoe, and F. Kydland (1992),
"International Real Business Cycles", Journal of Political Economy 100 (4):
745--775.

\bibitem{} Bejan, M. (2006), \textquotedblleft Trade Openness and Output
Volatility,\textquotedblright\ manuscript,
http://mpra.ub.uni-muenchen.de/2759/.

\bibitem{} Bloom, Nicholas; Draca Mirko and John Van Reenen (2016):
\textquotedblleft Trade Induced Technical Change? The Impact of Chinese
Imports on Innovation, IT and Productivity,\textquotedblright\ \textit{The
Review of Economic Studies}, Volume 83, Issue 1, Pages 87--117.

\bibitem{} Broda, C. and D. Weinstein (2006), \textquotedblleft
Globalization and the Gains from Variety,\textquotedblright\ The Quarterly
Journal of Economics, MIT Press, vol. 121(2): 541-585, May.

\bibitem{} Burgess, R. and D. Donaldson (2012) \textquotedblleft Railroads
and the Demise of Famine in Colonial India,\textquotedblright\ MIT
manuscript.

\bibitem{} Burstein, A. and J. Vogel (2016), \textquotedblleft International
trade, technology, and the skill premium,\textquotedblright\ forthcoming
Journal of Political Economy.

\bibitem{} Burstein, A. and J. Cravino (2015), \textquotedblleft Measured
Aggregate Gains from International Trade\textquotedblright\ with Javier
Cravino, American Economic Journal: Macroeconomics, vol 7 (2): 181-218.

\bibitem{} Caliendo, L., M.Dvorkin, and F. Parro (2019) \textquotedblleft
Trade and Labor Market Dynamics: General Equilibrium Analysis of the China
Trade Shock,\textquotedblright\ Econometrica, 87(3), 741-835.

\bibitem{} Caliendo, L. and F. Parro (2015) \textquotedblleft Estimates of
the Trade and Welfare Effects of NAFTA,\textquotedblright\ Review of
Economic Studies, 82(1), 1-44.

\bibitem{} Caliendo, L., F. Parro, E. Rossi-Hansberg and D. Sarte (2014).
\textquotedblleft The impact of regional and sectoral productivity changes
on the U.S. economy,\textquotedblright\ Princeton and Yale manuscripts.

\bibitem{} Cavallo, E. (2008). \textquotedblleft Output Volatility and
Openess to Trade: a Reassessment,\textquotedblright\ Journal of LACEA
Economia, Latin America and Caribbean Economic Association.

\bibitem{} Costello, D. (1993) \textquotedblleft A Cross-Country,
Cross-Industry Comparison of Productivity Growth,\textquotedblright\ Journal
of Political Economy, Vol. 101(2): 207-222.

\bibitem{} Costinot, A., D. Donaldson, and I. Komunjer (2012):
\textquotedblleft What Goods Do Countries Trade?A Quantitative Exploration
of Ricardo's Ideas,\textquotedblright\ Review of Economic Studies, 79,
581-608.

\bibitem{} Department for International Development (2011),
\textquotedblleft Economic openness and economic prosperity: trade and
investment analytical paper\textquotedblright\ (2011), prepared by the U.K.
Department of International Development's Department for Business,
Innovation \& Skills, February 2011.

\bibitem{} di Giovanni, J. and A. Levchenko (2009). \textquotedblleft Trade
Openness and Volatility,\textquotedblright\ The Review of Economics and
Statistics, MIT Press, vol. 91(3): 558-585, August.

\bibitem{} di Giovanni, J. and A. Levchenko (2012), \textquotedblleft
Country Size, International Trade, and Aggregate Fluctuations in Granular
Economies,\textquotedblright\ Journal of Political Economy, 120 (6):
1083-1132.

\bibitem{} di Giovanni, J, A. Levchenko and I. Mejean (2014),
\textquotedblleft Firms, Destinations, and Aggregate
Fluctuations,\textquotedblright\ Econometrica, 82:4, pages 1303-1340.

\bibitem{} di Giovanni, J., A. Levchenko, and J. Zhang (2014).
\textquotedblleft The Global Welfare Impact of China: Trade Integration and
Technological Change,\textquotedblright\ American Economic Journal:
Macroeconomics.

\bibitem{} Donaldson, D. \textquotedblleft Railroads of the Raj: Estimating
the Impact of Transportation Infrastructure,\textquotedblright\ (2015)
forthcoming, American Economic Review.

\bibitem{} Easterly, W., R. Islam, and J. Stiglitz (2001), \textquotedblleft
Shaken and Stirred: Explaining Growth Volatility,\textquotedblright\ Annual
World Bank Conference on Development Economics, p. 191-212. World Bank,
July, 2001.

\bibitem{} Eaton, J. and S. Kortum (2002), \textquotedblleft Technology,
Geography and Trade,\textquotedblright\ Econometrica 70: 1741-1780.

\bibitem{} Frankel, J. and A. Rose (1998), \textquotedblleft The Endogeneity
of the Optimum Currency Area Criteria,\textquotedblright\ Economic Journal,
Vol. 108, No. 449 (July):. 100-120.

\bibitem{} Haddad, M., J. Lim, and C. Saborowski (2013), \textquotedblleft
Trade Openness Reduces Growth Volatility When Countries Are Well
Diversified,\textquotedblright\ Canadian Journal of Economics, 46(2), 765-90.

\bibitem{} Head, K. and J. Ries (2001), \textquotedblleft Increasing Returns
versus National Product Differentiation as an Explanation for the Pattern of
U.S.-Canada Trade.\textquotedblright\ American Economic Review 91: 858-876.

\bibitem{} Hsieh, C. and Ossa, R. (2011), \textquotedblleft A Global View of
Productivity Growth in China,\textquotedblright\ \textit{Journal of
International Economics} 102: 209-224, September 2016.

\bibitem{} Kehoe, T. and K. J. Ruhl (2008), \textquotedblleft Are Shocks to
the Terms of Trade Shocks to Productivity?,\textquotedblright\ Review of
Economic Dynamics, Elsevier for the Society for Economic Dynamics, vol.
11(4): 804-819, October.

\bibitem{} Koren, M. and S. Tenreyro (2007), \textquotedblleft Volatility
and Development,\textquotedblright\ Quarterly Journal of Economics, 122 (1):
243-287.

\bibitem{} Koren, M. and S. Tenreyro (2013), \textquotedblleft Technological
Diversification,\textquotedblright\ The American Economic Review, February
2013, Volume 103(1): 378-414.

\bibitem{} Kose, A., E. Prasad, and M. Terrones (2003), \textquotedblleft
Financial Integration and Macroeconomic Volatility,\textquotedblright\ IMF
Staff Papers, Vol 50, Special Issue, p. 119-142.

\bibitem{} Kose, A. and K. Yi, (2001), \textquotedblleft International Trade
and Business Cycles: Is Vertical Specialization the Missing
Link?,\textquotedblright\ American Economic Review, vol. 91(2): 371-375, May.

\bibitem{} Levchenko, A. and J. Zhang (2013), \textquotedblleft The Global
Labor Market Impact of Emerging Giants: a Quantitative
Assessment,\textquotedblright\ IMF Economic Review, 61:3 (August 2013),
479-519.

\bibitem{} Levchenko, A. and J. Zhang (2014), \textquotedblleft Ricardian
Productivity Differences and the Gains from Trade,\textquotedblright\
European Economic Review, 65 (January 2014), 45-65.

\bibitem{} Levchenko, A. and J. Zhang (2016), \textquotedblleft The
Evolution of Comparative Advantage: Measurement and Welfare
Implications,\textquotedblright\ Journal of Monetary Economics, 78, 96-111

\bibitem{} Nakamura, Emi, and Jón Steinsson. 2008. “Five Facts about Prices: A Reevaluation of Menu Cost Models.” The Quarterly Journal of Economics 123 (4): 1415–64.

\bibitem{} Nakamura, Emi, and Jón Steinsson. 2012. “Lost in Transit: Product Replacement Bias and Pricing to Market.” The American Economic Review 102 (7): 3277–3316.

\bibitem{} Newbery, D. and J. Stiglitz, (1984), \textquotedblleft Pareto
Inferior Trade,\textquotedblright\ Review of Economic Studies, Wiley
Blackwell, vol. 51(1): 1-12, January.

\bibitem{} Parinduri, R. (2011), \textquotedblleft Growth Volatility and
Trade: Evidence from the 1967-1975 Closure of the Suez
Canal,\textquotedblright\ manuscript University of Nottingham.

\bibitem{} Parro, F. (2013), \textquotedblleft Capital-Skill Complementarity
and the Skill Premium in a Quantitative Model of Trade,\textquotedblright\
American Economic Journal: Macroeconomics, American Economic Association,
vol. 5(2): 72-117, April.

\bibitem{} Rodrik, D., (1998), \textquotedblleft Why Do More Open Economies
Have Bigger Governments?,\textquotedblright\ Journal of Political Economy,
vol. 106(5): 997-1032, October.

\bibitem{} Simonovska, I. and M. E. Waugh (2014), \textquotedblleft The
Elasticity of Trade: Estimates \& Evidence,\textquotedblright\ Journal of
International Economics, 92(1), 34-50.

\bibitem{} Stockman, A. (1988), \textquotedblleft Sectoral and National
Aggregate Disturbances to Industrial Output in Seven European
Countries,\textquotedblright\ Journal of Monetary Economics 21 (March):
387-409.

\bibitem{} Buch, C., J. D\"{o}pke and H. Strotmann (2009), \textquotedblleft
Does trade openness increase firm-level volatility?,\textquotedblright\
World Economy.

\bibitem{} Wacziarg, R. and J. S. Wallack (2004), \textquotedblleft Trade
liberalization and intersectoral labor movements,\textquotedblright\ Journal
of International Economics 64 (2004) 411-- 439.
\end{thebibliography}

\section*{Appendix}

\subsection*{Derivation of national income under free trade}

In the one-sector economy, under free trade, prices are equalized across
countries. 
\begin{equation*}
P_{t}=P_{nt}=\left( \xi B\right) ^{1/\beta }\left\{
\sum_{m=1}^{N}T_{m}\left( A_{mt}\right) ^{\theta }\left( w_{mt}\right)
^{-\beta \theta }\right\} ^{\frac{-1}{\beta \theta }}
\end{equation*}%
Thus, from $d_{nmt}=\left( \xi B\right) ^{-\theta }T_{m}\left( A_{mt}\right)
^{\theta }\left( w_{mt}\right) ^{-\beta \theta }\left( P_{mt}\right) ^{\beta
\theta }$ we obtain: 
\begin{equation*}
d_{mnt}=T_{n}\left( A_{nt}\right) ^{\theta }\left( w_{nt}\right) ^{-\beta
\theta }\left\{ \sum_{m=1}^{N}T_{m}\left( A_{mt}\right) ^{\theta }\left(
w_{mt}\right) ^{-\beta \theta }\right\} ^{-1}
\end{equation*}%
and from $w_{nt}L_{nt}=\sum\nolimits_{m=1}^{N}d_{mnt}w_{mt}L_{mt},$, we
have: 
\begin{equation*}
w_{nt}=\left( \frac{T_{n}\left( A_{nt}\right) ^{\theta }}{L_{nt}}\right) ^{%
\frac{1}{1+\beta \theta }}V_{t}
\end{equation*}%
where $V_{t}\equiv \left[ \sum\nolimits_{m=1}^{N}\frac{w_{mt}L_{mt}}{%
\sum_{i=1}^{N}T_{i}\left( A_{it}\right) ^{\theta }\left( w_{it}\right)
^{-\beta \theta }}\right] ^{\frac{1}{1+\beta \theta }}$ is common to all
countries. Therefore, using the definition of $Z_{nt}$, and recalling our
definition of real income, $Y_{nt}=\frac{w_{nt}L_{nt}}{P_{nt}}$, we have%
\begin{align*}
Y_{nt}& =L_{nt}\left( \frac{T_{n}\left( A_{nt}\right) ^{\theta }}{L_{nt}}%
\right) ^{\frac{1}{1+\beta \theta }}V_{t}\left( \xi B\right) ^{1/\beta
}\left\{ \sum_{i=1}^{N}T_{i}\left( A_{it}\right) ^{\theta }\left( \left( 
\frac{T_{i}\left( A_{it}\right) ^{\theta }}{L_{it}}\right) ^{\frac{1}{%
1+\beta \theta }}V_{t}\right) ^{-\beta \theta }\right\} ^{\frac{1}{\beta
\theta }} \\
& =\left( \xi B\right) ^{1/\beta }\left( T_{n}A_{nt}^{\theta }L_{nt}^{\beta
\theta }\right) ^{\frac{1}{1+\beta \theta }}\left[ \sum_{i=1}^{N}\left(
T_{i}\left( A_{it}\right) ^{\theta }L_{it}^{\beta \theta }\right) ^{\frac{1}{%
1+\beta \theta }}\right] ^{\frac{1}{\beta \theta }} \\
& =\left( \xi B\right) ^{1/\beta }Z_{nt}^{\frac{1}{1+\beta \theta }}\left(
\sum_{m=1}^{N}Z_{mt}^{\frac{1}{1+\beta \theta }}\right) ^{\frac{1}{\beta
\theta }}
\end{align*}

\subsection*{Numerical Procedure for Model Equilibrium}

We use nested iterations to compute the model equilibrium.\textbf{\ Miklos I
changed the outer loop to the best of my understanding but I thunk changes
are needed to Inner Loop and Middle Loop as well.}

\subsubsection*{Inner Loop}

For a given pair of sectoral resource allocation $(L_{nt}^{j})$ and sectoral
wages $(w_{nt}^{j})$ solve the system below for sectoral price indexes $%
P_{nt}^{j}.$%
\begin{align*}
P_{nt}& =\prod_{j=1}^{J}{\alpha _{t}^{j}}^{-\alpha _{t}^{j}}{P_{nt}^{j}}%
^{\alpha _{t}^{j}} \\
P_{nt}^{j}& =\xi {\Phi _{nt}^{j}}^{-\frac{1}{\theta }} \\
\Phi _{nt}^{j}& ={B^{j}}^{-\theta }\sum_{i=1}^{N}T_{i}^{j}{A_{it}^{j}}%
^{\theta }\left( \frac{{w_{it}^{j}}^{\beta ^{j}}\prod_{k=1}^{J}{P_{it}^{k}}%
^{\gamma ^{kj}}}{\kappa _{nit}^{j}}\right) ^{-\theta }
\end{align*}

Simplify $\Phi_{nt}^j:$ 
\begin{eqnarray*}
\Phi_{nt}^j &=& {B^j}^{-\theta} \sum_{i = 1}^N \underbrace{T_i^j {A_{it}^j}%
^{\theta}}_{\frac{Z_{it}^j}{{L_{it}}^{\beta^j\theta}}} \left(\frac{ {w_{it}^j%
}^{\beta^j} \prod_{k = 1}^{J}{P_{it}^k}^{\gamma^{kj}}}{\kappa_{nit}^j}%
\right)^{-\theta} \\
&=& {B^j}^{-\theta} \sum_{i = 1}^N Z_{it}^j {L_{it}}^{- \beta^j\theta} {%
w_{it}^j}^{-\beta^j\theta} {\kappa_{nit}^j}^{\theta} \prod_{k = 1}^{J}{%
P_{it}^k}^{-\theta\gamma^{kj}} \\
&=& {B^j}^{-\theta} \sum_{i = 1}^N \underbrace{Z_{it}^j {%
\left((L_{it}w_{it}^j)^{- \beta^j} \kappa_{nit}^j\right)}^{\theta}}%
_{D_{nit}^j} \prod_{k = 1}^{J}{P_{it}^k}^{-\theta\gamma^{kj}} \\
&=& {B^j}^{-\theta} \sum_{i = 1}^N D_{nit}^j \prod_{k = 1}^{J}{P_{it}^k}%
^{-\theta\gamma^{kj}}
\end{eqnarray*}
Note that we can compute the coefficients of the equation (the $D$ values)
before starting the search for prices. Now we can write the system of
equations as 
\begin{align*}
{P_{nt}^j}^{-\theta} &= \xi^{-\theta} {\Phi_{nt}^j} \\
{P_{nt}^j}^{-\theta} &= \xi^{-\theta} {{B^j}^{-\theta} \sum_{i = 1}^N
D_{nit}^j \prod_{k = 1}^{J}{P_{it}^k}^{-\theta\gamma^{kj}}}
\end{align*}
or 
\begin{equation}  \label{inner:loop}
\mathcal{P}_{nt}^j = \left(\xi B^j\right)^{-\theta} \sum_{i = 1}^N D_{nit}^j
\prod_{k = 1}^{J}{\mathcal{P}_{it}^k}^{\gamma^{kj}}
\end{equation}
where $\mathcal{P}_{nt}^j \equiv {P_{nt}^j}^{-\theta}$. We iterate %
\eqref{inner:loop} until convergence.

\subsubsection*{Middle loop}

For a given resource allocation, $L_{nt}^{j}$, this loop searches for
sectoral wages $w_{nt}^{j}$ that solve the system of equations below. For
notational simplicity we solve the system in terms of sectoral revenue and
then calculate corresponding wages from $w_{nt}^{j}L_{nt}^{j}=\beta
^{j}R_{nt}^{j}$. 
\begin{align*}
R_{nt}^{j}& =\sum_{m=1}^{N}E_{mt}^{j}d_{mnt}^{j}\!\left( w_{nt}^{j}\right) \\
E_{mt}^{j}& =\alpha _{t}^{j}P_{mt}C_{mt}+\sum_{k=1}^{J}\gamma ^{jk}R_{mt}^{k}
\\
P_{mt}C_{mt}& =\sum_{j=1}^{J}\beta ^{j}R_{mt}^{j}-S_{mt},
\end{align*}%
where $S_{mt}$ is the exogenous trade surplus of country $m$ in year $t$,
with $\sum_{m}S_{mt}=0$. In the baseline specification, we set $S_{mt}=0$
for each country. (A subsequent appendix explains how the trade surplus
enters into the numerical algorithm).

Substituting in, we get a system of linear equations in $R_{nt}^{j}$ for any
given value of $d_{mnt}^{j}(w_{nt}^{j})$: 
\begin{equation}
R_{nt}^{j}=\sum_{m=1}^{N}d_{mnt}^{j}(w_{nt}^{j})\left[ \alpha
_{t}^{j}\sum_{k=1}^{J}\beta ^{k}R_{mt}^{k}-\alpha
_{t}^{j}S_{mt}+\sum_{k=1}^{J}\gamma ^{jk}R_{mt}^{k}\right]
\label{middle:loop}
\end{equation}

Note that $d^j_{mnt}$ depends on $w_{nt}^j$ through 
\begin{equation*}
d_{mnt}^j = \frac{{B^j}^{-\theta} T_n^j {A_{nt}^j}^{\theta} \left(\frac{{%
w_{nt}^j}^{\beta^j} \prod_{k = 1}^{J}{P_{nt}^k}^{\gamma^{kj}} }{%
\kappa_{mnt}^j}\right)^{-\theta}}{{B^j}^{-\theta} \sum_{i = 1}^N T_i^j {%
A_{it}^j}^{\theta} \left(\frac{{w_{it}^j}^{\beta^j} \prod_{k = 1}^{J}{%
P_{it}^k}^{\gamma^{kj}} }{\kappa_{mit}^j}\right)^{-\theta}},
\end{equation*}
where prices were solved for in the inner loop. To facilitate computation we
introduce $D$, the coefficients from the inner loop. We can rewrite the
definition of $d$ as 
\begin{align*}
d_{mnt}^j &= \frac{\overbrace{T_n^j {A_{nt}^j}^{\theta}}^{\frac{Z_{nt}^j}{{%
L_{nt}}^{\beta^j\theta}}} \left(\frac{{P_{nt}}^{1 - \beta^j} {w_{nt}^j}%
^{\beta^j}}{\kappa_{mnt}^j}\right)^{-\theta}}{\sum_{i = 1}^N \underbrace{%
T_i^j {A_{it}^j}^{\theta}}_{\frac{Z_{it}^j}{{L_{it}}^{\beta^j\theta}}} \left(%
\frac{{P_{it}}^{1 - \beta^j} {w_{it}^j}^{\beta^j}}{\kappa_{mit}^j}%
\right)^{-\theta}} \\
&= \frac{Z_{nt}^j {L_{nt}}^{- \beta^j\theta} {w_{nt}^j}^{-\beta^j\theta} {%
\kappa_{mnt}^j}^{\theta}{P_{nt}}^{\theta(\beta^j - 1)}}{\sum_{i = 1}^N
Z_{it}^j {L_{it}}^{- \beta^j\theta} {w_{it}^j}^{-\beta^j\theta} {%
\kappa_{mit}^j}^{\theta}{P_{it}}^{\theta(\beta^j - 1)}} \\
&= \frac{\overbrace{Z_{nt}^j {L_{nt}}^{- \beta^j\theta} {w_{nt}^j}%
^{-\beta^j\theta} {\kappa_{mnt}^j}^{\theta}}^{D_{mnt}^j}{P_{nt}}%
^{\theta(\beta^j - 1)}}{\sum_{i = 1}^N \underbrace{Z_{it}^j {L_{it}}^{-
\beta^j\theta} {w_{it}^j}^{-\beta^j\theta} {\kappa_{mit}^j}^{\theta}}%
_{D_{mit}^j}{P_{it}}^{\theta(\beta^j - 1)}} \\
&= \frac{D_{mnt}^j {P_{nt}}^{\theta(\beta^j - 1)}}{\sum_{i = 1}^N D_{mit}^j{%
P_{it}}^{\theta(\beta^j - 1)}}
\end{align*}
Note that $d$ does not depend on the resource allocation.

We guess a wage to compute $d$, then solve the linear equations %
\eqref{middle:loop} for revenue. Our new wage can be computed from the
revenue, and we iterate until convergence.

\subsubsection*{Outer loop}

The goal of this loop is to find the sectoral resource allocations $%
L_{nt}^{j}$ that satisfy %
\begin{equation*}
\frac{L_{nt}^{j}}{L_{nt}}=E_{t-1}\left( \frac{w_{nt}^{j}L_{nt}^{j}}{%
w_{nt}L_{nt}}\right)
\end{equation*}%
where $w_{nt}$ is the average wage. When searching for the
equilibrium value of $L_{nt}^{j}$ the state of the economy is made
up of the deterministic component of the augmented-productivity processes, $%
\bar{Z}_{nt}^{j}$, as well as the previous-period values of the
log-deviation processes representing country, sector, and idyosincratic
shocks, $\lambda _{t-1}^{j}$, $\mu _{nt-1}$, and $%
\epsilon _{nt-1}^{j}$. This state is known both to us and to the
decision maker in the model, as are the autoregressive parameters driving the
shock processes and their variances. Hence, we can draw values from the
distribution of $\lambda _{t}^{j}$, $\mu _{nt}$, and $%
\epsilon _{nt}^{j}$ and combine them with $\bar{Z}_{nt}^{j}$
to create corresponding draws for $Z_{nt}^{j}$. For each iteration
over possible candidates for the $L_{nt}^{j}$s, we thus draw 100
random realizations of the $Z_{nt}^{j}$s, and for each of them we
compute $w_{nt}^{j}$ and $w_{nt}$, and hence $\frac{%
w_{nt}^{j}L_{nt}^{j}}{w_{nt}L_{nt}}$ from the middle loop. Then the
expectation of these wage shares is simply the average across all the draws
of $Z_{nt}^{j}$. The iteration ends when the left hand side and
right hand side are close enough.

\subsection*{Data Sources}

We first describe the sample of countries and then the various sources of
data.

\subsubsection*{Sample of Countries}

Our sample consists of 24 core countries, for which we were able to collect
all the information needed to carry out the quantitative analysis with no
need---or very limited need---of estimation. Other countries, for which data
are nearly complete and estimation of some sectors' output or value added
was needed, are grouped as \textquotedblleft Rest of the
World\textquotedblright\ (ROW); the sectoral trade data are available for
virtually all countries. Some countries were aggregated (for example Belgium
and Luxembourg, and, before making it into ROW, Former USSR, Former
Yugoslavia.). In particular, the minimum condition to keep a country (or an
aggregation of countries) in the sample is the availability of complete
series of sectoral value added and the presence of trade data.

The core sample of countries include the United States, Mexico, Canada,
Australia, China, Japan, South Korea, India, Colombia, the United Kingdom, a
composite of France and its overseas departments, Germany, Italy, Spain,
Portugal, a composite of Belgium and Luxembourg, the Netherlands, Finland,
Sweden, Norway, Denmark, Greece, Austria and Ireland. While some important
countries appear only in our ROW group (most notably Brazil, Russia, Turkey,
Indonesia, Malaysia and oil exporters), the selection of core countries is
meaningful both in terms of geographic location (covering all continents)
and in terms of their share in global trade and GDP. The time period we
study covers years from 1972 to 2007.\ 1970--1971 are slightly problematic
for trade data, as there are many missing observations; hence the decision
to start in 1972. The end period is chosen in order to avoid confounding the
trade effects we are after with the financial crisis, which had other
underlying causes. We focus on annual data.

\subsubsection*{Sectoral Gross Output}

The data are disaggregated into 24 sectors: agriculture (including mining
and quarrying), 22 manufacturing sectors, and services, all available in US
dollars for the core countries and the Rest of the world (ROW). The 22
manufacturing sectors correspond to the industries numbered 15 to 37 in the
ISIC Rev. 3 classification (36 and 37 are bundled together).

The final dataset is obtained by combining different sources and some
estimation. Data on agriculture, aggregate manufacturing, and services for
core countries come mostly from the EU KLEMS database. There is no available
series for services output in China and India, so they are obtained as
residuals. Additional data come from the UN National Accounts.

Data on manufacturing subsectors come from UNIDO and EU KLEMS. For some
subsectors, EU KLEMS data are available only at a higher level of
aggregation (i.e. sector 15\&16 instead of the two separately); in those
cases, we use the country specific average shares from UNIDO for the years
in which they are available to impute values for each subsectors.

For the countries in the ROW, the output dataset is completed through
estimation, using sectoral value added, aggregate output, GDP and population
(the latter two from the Penn World Table 7.1) in a Poisson regressions.

Finally, for the few countries for which we have sectoral value added data
(described below) but no PWT data, we estimate sectoral output by
calculating for each year and sector the average value added/output ratio, 
\begin{equation*}
\bar{\beta}_{t}^{j}=\frac{1}{N}\sum_{i=1}^{N}\frac{VA_{i,t}^{j}}{%
Output_{i,t}^{j}}
\end{equation*}%
and then use it in 
\begin{equation*}
\widehat{Output_{i,t}^{j}}=\frac{VA_{i,t}^{j}}{\bar{\beta}_{t}^{j}}
\end{equation*}%
Data collection notes on the core countries are as follows:

\begin{itemize}
\item USA: missing years 1970-76 generated using a growth rate of each
sector from EU KLEMS (March 2008 edition).

\item Canada: 1970-04 EU KLEMS (March 2008 edition), for 2005-06 sectoral
growth rates from the Canadian Statistical Office's National Economic
Accounts (table Provincial gross output at basic prices by industries).

\item China: data are from the Statistical yearbooks of China. Output in
agriculture is defined as gross output value of farming, forestry, animal
husbandry and fishery and is available for all years. Mining and
manufacturing is reported as a single unit labelled output in industry,
which apart from the extraction of natural resources and manufacture of
industrial products includes sectors not covered by other countries: water
and gas production, electricity generation and supply and repair of
industrial products (no adjustment was made). The primary concern was the
methodological change initiated around 1998, when China stopped reporting 
\textit{total} industrial output and limited the coverage to industrial
output of firms with annual sales above 5m yuan (USD 625 000). The sectoral
coverage remained the same in both series. There were 5 years of overlapping
data of both series over which the share of the 5m+ firms on total output
decreased from 66 to 57 percent. The chosen approach to align both series
was to take the levels of output from the pre-1999 series (output of all
firms) and apply the growth rate of output of 5m+ firms in the post-1999
period. This procedure probably exaggerates the level of output in the last
seven years and leads to an enormous increase in the output/GDP in industry
ratio (from 3.5 in 1999 to 6.0 in 2006). Our conjecture is that the ratio
would be less steep if the denominator was value added in industry
(unavailable on a comparable basis) because the GDP figure includes net
taxes, which might take large negative values. Output in industry of all
firms reflects the 1995 adjustment with the latest economic census.

There is no available estimate for output in services, so we use the
predicted values from a Poisson regression on the other core countries, with
sectoral value added (see below for details on the source), output in
agriculture, output in manufacturing, GDP and population (the latter two
from the Penn World Table 7.1) and year dummies as regressors.

\item India: data are from the Statistical Office of India, National
Accounts Statistics. Years 1999-06 are reported on the SNA93 basis. Earlier
years were obtained using the growth rates of sectoral output as defined in
their `Back Series' database. The main issue with India was the large share
of `unregistered' manufacturing that is reported in the SNA93 series but
missing in the pre-1999 data. The `unregistered' manufacturing covers firms
employing less than 10 workers and is also referred to as the informal or
unorganized sector. We reconstructed the total manufacturing output using
the assumption that the share of registered manufacturing output in total
manufacturing output mirrors the share of value added of the registered
manufacturing sector in total value added in manufacturing (available from
the `Back Series' database).

As for China, output in services was estimated through a Poisson regression
method.

\item Mexico: data are from the System of National Accounts published by the
INEGI and from the UN National Accounts Database. 2003-06 Sistema de cuentas
nacionales, INEGI (NAICS), 1980-03 growth rate from the UN National Accounts
Data, 1978-79 growth rate from Sistema de cuentas nacionales, INEGI,
1970-1978 growth rate from System of National Accounts (1981), Volumen I
issued by the SPP.

\item Japan: data for 1973-06 are from EU KLEMS (November 2009 Edition), for
1970-72 the source is the OECD STAN database (growth rate).

\item Colombia and Norway: data are from the UN National Accounts Database.

\item Germany: the series is EU KLEMS' estimate for both parts of Germany.
\end{itemize}

The exchange rates used for the conversion of output data come from the IMF.

\subsubsection*{Sectoral Value Added}

The data on sectoral value added is obtained by combining data from the
World Bank, UN National Accounts, EU KLEMS and UNIDO. For the World Bank and
UN cases, the format of the data does not allow to have exactly the same
sectoral classification as the output data: namely, mining here is not
included in agriculture.

The World Bank and UN data are cleaned (we noted a contradiction in the UN
data for Ethiopia and Former Ethiopia, which we correct to include in ROW
final sample).

Data on manufacturing subsectors come from UNIDO and EU KLEMS. For some
subsectors, EU KLEMS data are available only at a higher level of
aggregation (i.e. sector 15\&16 instead of the two separately); in those
cases, we use the country specific average shares from UNIDO for the years
in which they are available to impute values for each subsectors; if no such
data are available in UNIDO, we use the average shares for the whole sample.
We use the UNIDO data as baseline and complete it with EU KLEMS when
necessary (in these cases the growth rates of the EU KLEMS series are used
to impute values; this is done because sometimes the magnitudes are quite
different in the two datasets). If an observation is missing in both
datasets, we impute it using the country specific average sectoral shares
for the years in which data are available.

\subsubsection*{Trade flows}

We use bilateral imports and exports at the sectoral level from 1972 to 2007
from the UN COMTRADE database. This dataset contains the value of all the
transactions with international partners reported by each country. Since
every transaction is potentially recorded twice (once reported by the
exporter and once by the importer) we use the values reported by the
importer when possible and integrate with the corresponding values reported
by the exporter if only those are available. Re-exports and re-imports are
not included in the exports and imports figures.

We use the SITC1 classification for all the sample. This is made in order to
ensure a consistent definition of the sectors throughout the whole time
period. In order to construct the agricultural sector we aggregate the
subsectors in the SITC1 classification corresponding to the BEC11 group. For
the manufacturing sectors, we use the correspondence tables available on the
UN website to identify the SITC1 groups corresponding to the ISIC 3 groups
used for output and value added.

\subsubsection*{Prices}

In order to back out the augmented productivity processes $Z_{nt}^{j}$ we
require aggregate price indices for all countries. For the resulting $%
Z_{nt}^{j}$ to be comparable across countries, these price indices must be
in a common currency. Hence, we use the price of GDP variable from the Penn
World Tables (PWT) which is expressed in a common unit (so-called
\textquotedblleft international dollars\textquotedblright ).\footnote{%
Strictly speaking a better match between the price of GDP in the model and
in the data would have been the price of consumption, but as is well known
these variables take almost identical values in the PWTs. It is important to
note that we use the PWT for $P_{nt}$ only in the procedure to back out the $%
Z_{nt}^{j}$s. As discussed later, when we compute real agregate income in
the data to generate aggregate volatility figures to compare to the model
output, we do not need to worry about having the prices in the same
currency, and we are therefore able to use national CPIs, which map exactly
into the theoretical counterpart.} In particular, we use version 7.1 of PWT
for all countries, except for Former USSR, Former Czechoslovakia and Former
Yugoslavia, for which we use the PWT 5.6. For the ROW, we compute a weighted
average of the relative prices of GDP for all the countries for which the
PWT data are available (most of the ROW countries), where the weights are
each country's share of total output. Similarly, for Belgium-Luxembourg, we
compute the weighed average of the two.

For the augmented productivity processes we also require sectoral price
deflators from the USA. These are taken from EU\ KLEMS.

\subsubsection*{Real Income}

We need a time series for real income to generate volatility figures to
compare to the volatility implied by our model. We use nominal value added
(the aggregate for all sectors) in local currency units, deflated by the
countries' CPI. The data are provided by the World Bank's World Development
Indicators, in turn sourced by the International Monetary Fund (IMF). For
Germany we use the CPI index provided by the OECD, as the IMF index is not
consistent over time. For the United Kingdom we use the Retail Price Index,
as the CPI index is not available.

\subsection*{Trade Imbalances}

In the presence of trade imbalances, equation (\ref{BC}) becomes%
\begin{equation*}
P_{nt}C_{nt}=\sum\nolimits_{j=1}^{J}w_{nt}^{j}L_{nt}^{j}-S_{nt},
\end{equation*}%
where $S_{nt}$ is the exogenously given current account surplus. As a
consequence, the first order condition for labor allocations becomes%
\begin{equation*}
\frac{L_{nt}^{j}}{L_{nt}}=\frac{E_{t-1}\left( \frac{w_{nt}^{j}L_{nt}^{j}}{%
\sum_{j=1}^{J}w_{nt}^{j}L_{nt}^{j}-S_{nt}}\right) }{E_{t-1}\left( \frac{%
\sum_{j=1}^{J}w_{nt}^{j}L_{nt}^{j}}{\sum_{j=1}^{J}w_{nt}^{j}L_{nt}^{j}-S_{nt}%
}\right) }.
\end{equation*}%
It can easily be shown that (\ref{eq:FOC}) is the first order approximation
of the expression above around $S_{nt}=0.$ Hence, there is no compelling
quantitative reason to change this part of the model when allowing for trade
imbalances.

On the other hand, equation (\ref{exp}) becomes%
\begin{equation*}
E_{mt}^{j}=\alpha _{t}^{j}\left( P_{mt}C_{mt}-S_{mt}\right)
+\sum\nolimits_{k=1}^{J}\gamma ^{jk}R_{mt}^{k}.
\end{equation*}%
Since $S_{mt}$ enters this linearly, the model must be solved again with
this equation instead of the original (\ref{exp}).

\end{document}
