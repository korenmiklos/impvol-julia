\documentclass[12pt]{article}
\usepackage{amsmath}
\begin{document}
\begin{equation}
	\frac \sigma{\sigma-1}
	E \ln\left[
	 \sum C_i^{1-1/\sigma}
	 \right] \to \max
\end{equation}
\[
\text{s.t. }
C_i = A_i L_i 
\]
The expectation is over productivity shocks, taking labor allocation as given. Labor demand across sectors has to add up to total fixed labor supply. We are in autarky.

Let $w=1$. Then real income is $P^{-1}$ and $P_i=A_i^{-1}$. How do productivity shocks affect the CES price index $P$?
\[
P = \left[\sum_i P_i^{1-\sigma}
\right]^{1/(1-\sigma)}
\]
so that
\[
U = \left[\sum_i A_i^{\sigma-1}
\right]^{1/(\sigma-1)}.
\]
Do a log-linear approximation so that we can apply the delta method for calculating variance
\[
d\ln U \approx
\sum_i U^{1-\sigma}
 A_i^{\sigma-1}
d\ln A_i=
\]
\[
\sum_i\frac{A_i^{\sigma-1}}
	{\sum_j A_j^{\sigma-1}}
d\ln A_i
\]
Suppose variance of $d\ln A_i=\Sigma$, iid across sectors. Then, by the delta method,
\[
\text{Var d}\ln U \approx \Sigma \sum_i s_i^2,
\]
where 
\[
s_i = \frac{A_i^{\sigma-1}}
	{\sum_j A_j^{\sigma-1}}.
\]
Clearly, when $\sigma=1$, $\sum s_i^2 = 1/n$ ,the lowest possible value for a Herfindahl index. If $\sigma>1$ or $\sigma<1$, the volatility is larger, because the cost share of sectors is more unequal. It also means there is more response of volatility to changes in the parameters across sectors.  
\end{document}